\section{Einleitung}
Einleitende Worte...
Kurze Vorstellung der Firma groninger ?
\subsection{Motivation / Problemstellung}
\label{sec:Motivation}
Im Rahmen der Industrie 4.0 spielen digitale Zwillinge eine zunehmend zentrale Rolle. 
Viele Unternehmen setzen bereits auf proprietäre Lösungen, bei denen ein digitales Abbild eines physischen Assets erschaffen wird, um Abläufe transparenter und effizienter zu gestalten. 
Mit der steigenden Nachfrage nach digitalen Zwillingen steigt jedoch auch der Bedarf an standardisierten und interoperablen Lösungen.
In den vergangenen Jahren hat sich die \ac{idta} intensiv mit dieser Thematik ausseinandergesetzt und als Lösung die sogenannte Verwaltungsschale (\ac{aas}) etabliert.
Diese ermöglicht eine interoperable Umsetzung digitaler Zwillinge, wobei sämtliche relevanten Informationen zu einem Asset über dessen gesamten Lebenszyklus hinweg digital erfasst, strukturiert und dokumentiert werden können. 
Immer mehr Unternehmen befassen sich mit dem Konzept der Verwaltungsschale, und einige haben es bereits in Form von Pilotprojekten, Showcases oder vereinzelt sogar schon in bestehende Systeme integriert.

Aber auch regulatorische Anforderungen wie der digitale Produktpass, der von der EU entwickelt und künftig für viele Produktgruppen verpflichtend wird, zeigen die Notwendigkeit von einheitlichen digitalen Zwillingen. 
Um diesen Anforderungen in Zukunft gerecht zu werden, ist es wichtig, sich schon frühzeitig mit der Verwaltungsschale ausseinanderzusetzen.
\subsection{Zielsetzung}
Ziel diser Arbeit ist die Modellierung eines digitalen Zwillings für eine Abfüllmaschine der Firma groninger.
Die Umsetzung erfolgt auf Basis der Verwaltungsschale unter Verwendung verschiedener Open Source Tools, die unter anderem von der \acs{idta} bereitgestellt werden. 
Im Fokus steht dabei das Abfüll -und Verschließmodul der robocell-Linie, welche sich durch einen hohen Automatisierungsgrad auszeichnet.
% Noch ein Satz zur robocell
Im Rahmen der Arbeit soll überprüft werden, ob der Einsatz der Verwaltungsschale hinsichtlich der Interoperabilität, Produktivität und Nachverfolgbarkeit einen Mehrwert für groninger bietet.
Ebenfalls sollen die Tools zur Modellierung eines solchen digitalen Zwillinges evaluiert werden, wobei zum Beispiel der AASX Package Explorer zur Erstellung oder ein Server zum Verwalten der Verwaltungsschalen genutzt werden soll.
Ebenfalls soll die Einsatzmöglichkeit von KI zur Optimierung diskutiert werden.
Abschließend soll eine Handlungsempfehlung für groninger formuliert werden, ob der Einsatz der Verwaltungsschale einen Mehrwert für das Unternehmen bieten kann.
\subsection{Vorgehensweise}
Die Arbeit beginnt mit einer eingehenden Betrachtung grundlegender Konzepte, die für das Verständnis der Arbeit notwendig sind. 
Dabei wird insbesondere die Frage geklärt, was die Verwaltungsschale und der digitale Zwilling überhaupt sind und wie diese sich im industriellen Kontext - insbesondere im Rahmen von Industrie 4.0 - einordnen.

Anschließend wird auf die Entwicklung des digitalen Zwillings der robocell eingegangen. 
Dabei werden verschiedene Bereiche der Maschine betrachtet und in Teilmodellen bzw. verschiedenen Submodellen modelliert.
Unter anderem wird ein Anwendungsfall für den digitalen Produktpass umgesetzt.
Aber auch das allgemeine Vorgehen bei der Erstellung eines digitalen Zwillings mit Hilfe der Verwaltungsschale wird aufgezeigt.
Dabei spielt unteranderem das automatische Erstellen und Bereitstellen von Verwaltungsschalen eine zentrale Rolle.
Für die technische Umsetzung kommen verschiedene Tools und Technologien zum Einsatz, wie zum Beispiel der AASX Package Explorer zum Erstellen oder BaSyx als AAS-Laufzeitumgebung zum Verwalten der verschiedenen Verwaltungsschalen.


Im Ergebnisteil wird der AAS-Demonstrator für die robocell vorgestellt. Es werden sowohl die Systemarchitektur als auch die modellierten Teilmodelle präsentiert.
Aber auch Herausforderungen bei der Implementierung werden aufgezeigt.
Zudem werden Potenziale des Einsatz von KI im Zusammenhang mit der Verwaltungsschale aufgezeigt und diskutiert.
Außerdem werden die eingesetzen Open Source Lösungen zur Modellierung und Verwaltung der Verwaltungsschale evaluiert.


Es folgt eine Zusammenfassung der Arbeit, bei der die wichtigsten Aspekte kurz zusammengefasst werden.
Zudem wird eine Handlungsempfehlung für groninger formuliert und ein Ausblick auf zukünftige Entwicklungen geliefert.