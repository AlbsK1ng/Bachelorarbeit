\section{Einleitung}
\subsection{Motivation}
\label{sec:Motivation}
Im Zuge der vierten industriellen Revolution gewinnen digitale Zwillinge zunehmend an Bedeutung. 
Sie ermöglichen die digitale Abbildung physischer Assets und schaffen damit die Grundlage für transparentere und effizientere industrielle Prozesse.
Viele Unternehmen setzen bereits auf solche digitalen Abbilder, verwenden dabei jedoch häufig proprietäre Lösungen, die nicht interoperabel sind und somit den Austausch von Daten über System- und Unternehmensgrenzen hinweg erschweren. 

Mit der steigenden Nachfrage nach digitalen Zwillingen wächst auch der Bedarf an standardisierten und herstellerunabhängigen Lösungen. 
In den vergangenen Jahren hat sich die Plattform Industrie 4.0 intensiv mit dieser Thematik auseinandergesetzt und als Antwort auf diese Herausforderung die sogenannte \ac{aas} etabliert.
Sie stellt ein standardisiertes Rahmenwerk zur interoperablen Umsetzung digitaler Zwillinge dar, mit dem sämtliche relevanten Informationen zu einem Asset über dessen gesamten Lebenszyklus hinweg digital erfasst, strukturiert und dokumentiert werden können.

Die zunehmende Beschäftigung zahlreicher Unternehmen mit dem Konzept der \acs{aas} sowie die Umsetzung in Pilotprojekten und ersten produktiven Anwendungen unterstreichen die hohe Relevanz und das Potenzial dieser Lösung. 
Gleichzeitig verdeutlichen regulatorische Anforderungen, wie der digitale Produktpass (\acs{dpp}), der von der EU entwickelt und künftig für viele Produktgruppen verpflichtend sein wird, die Notwendigkeit eines einheitlichen Konzepts zur standardisierten und semantischen Beschreibung von Produkten.

Um den wachsenden Anforderungen an Transparenz, Rückverfolgbarkeit und Effizienz gerecht zu werden und technologisch auf dem neuesten Stand zu bleiben, ist es für Unternehmen daher unerlässlich, sich bereits frühzeitig mit der \acs{aas} auseinanderzusetzen.

\subsection{Zielsetzung}
Ziel dieser Arbeit ist die prototypische Modellierung eines \acs{aas}-basierten digitalen Zwillings für das Abfüll- und Verschließmodul der robocell-Linie der Firma groninger. 
Dabei soll untersucht werden, inwiefern der Einsatz dieses Konzepts dem Unternehmen einen Mehrwert bietet, insbesondere im Hinblick auf Interoperabilität, Produktivität und Nachverfolgbarkeit.

Unter Interoperabilität wird dabei die herstellerunabhängige und standardisierte Kommunikation zwischen Maschinen und Systemen verstanden, die eine nahtlose Vernetzung in einem Unternehmen als auch über Unternehmensgrenzen hinweg ermöglicht.
Hinsichtlich der Produktivität steht die effiziente Bereitstellung und der Austausch von Informationen im Fokus, mit Potenzialen zur Optimierung von Prozessen, etwa durch vorausschauende Instandhaltungsmaßnahmen. 
Die Nachverfolgbarkeit schließlich umfasst die lückenlose Erfassung und Rückverfolgung aller relevanten Asset-Daten über den gesamten Lebenszyklus, wodurch Zustandsänderungen und Ereignisse transparent und nachvollziehbar werden.

Die Modellierung sowie die Bereitstellung in einem Industrie 4.0-Ökosystem soll mithilfe verschiedener Open-Source-Tools erfolgen, die unter anderem von der \ac{idta} bereitgestellt werden. 
Zudem soll ein KI-gestütztes Konzept zur Optimierung anhand der im digitalen Zwilling abgebildeten Daten evaluiert und dessen Umsetzung aufgezeigt werden. 
Auf Grundlage der gewonnen Erkenntnisse soll abschließend eine Handlungsempfehlung für groninger formuliert werden, die den möglichen Nutzen und die Relevanz des Einsatzes der \acs{aas} für das Unternehmen bewertet.

\subsection{Vorgehensweise}
Die Arbeit beginnt mit einer eingehenden Betrachtung grundlegender Konzepte, die für das Verständnis der Arbeit notwendig sind. 
Dabei wird insbesondere die Frage geklärt, was die \acs{aas} und der digitale Zwilling überhaupt sind und wie diese sich im industriellen Kontext, insbesondere im Rahmen von Industrie 4.0, einordnen.

Anschließend wird auf die Entwicklung des digitalen Zwillings des Abfüll -und Verschließmoduls der robocell eingegangen. 
Dabei werden verschiedene Bereiche der Maschine betrachtet und in verschiedenen Submodellen der \acs{aas} modelliert.
Unter anderem wird ein Anwendungsfall für den digitalen Produktpass sowie das automatisierte Erstellen der \acs{aas} umgesetzt.
Aber auch das allgemeine Vorgehen bei der Erstellung eines digitalen Zwillings mithilfe der Verwaltungsschale wird aufgezeigt.
Für die technische Umsetzung kommen verschiedene Tools und Technologien zum Einsatz, wie zum Beispiel der AASX Package Explorer zum Erstellen oder BaSyx als \acs{aas}-Laufzeitumgebung zum Verwalten der verschiedenen Verwaltungsschalen.

Im Ergebnisteil wird der \acs{aas}-Demonstrator für die robocell vorgestellt. Es werden sowohl die Systemarchitektur als auch die modellierten Teilmodelle präsentiert.
Aber auch Herausforderungen bei der Implementierung werden beleuchtet.
Zudem wird diskutiert, welche Potenziale \ac{ki} im Zusammenhang mit der \acs{aas} bietet.
Außerdem werden die eingesetzen Open-Source-Lösungen zur Modellierung und Verwaltung der AAS evaluiert.

Es folgt eine Zusammenfassung der Arbeit, bei der die wichtigsten Aspekte kurz zusammengefasst werden.
Zudem wird eine Handlungsempfehlung für groninger formuliert und ein Ausblick auf zukünftige Entwicklungen geliefert.