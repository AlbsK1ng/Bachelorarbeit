\thispagestyle{fancy}
\section{Einleitung}
\subsection{Motivation}
\label{sec:Motivation}
Im Zuge der vierten industriellen Revolution gewinnen digitale Zwillinge zunehmend an Bedeutung. 
Sie ermöglichen die digitale Abbildung physischer Assets und schaffen damit die Grundlage für transparentere und effizientere industrielle Prozesse.
Viele Unternehmen setzen bereits auf solche digitalen Abbilder, verwenden dabei jedoch häufig proprietäre Lösungen, die nicht interoperabel sind und somit den Austausch von Daten über System- und Unternehmensgrenzen hinweg erschweren. 

Mit der steigenden Nachfrage nach digitalen Zwillingen wächst auch der Bedarf an standardisierten und herstellerunabhängigen Lösungen. 
In den vergangenen Jahren hat sich die Plattform Industrie 4.0 intensiv mit dieser Thematik auseinandergesetzt und als Antwort auf diese Herausforderung die sogenannte \ac{aas} etabliert.
Sie stellt ein standardisiertes Rahmenwerk zur interoperablen Umsetzung digitaler Zwillinge dar, mit dem sämtliche relevanten Informationen zu einem Asset über dessen gesamten Lebenszyklus hinweg digital erfasst, strukturiert und dokumentiert werden können.

Die zunehmende Beschäftigung zahlreicher Unternehmen mit diesem Konzept sowie die Umsetzung in Pilotprojekten und ersten produktiven Anwendungen unterstreichen die hohe Relevanz und das Potenzial dieser Lösung. 
Gleichzeitig verdeutlichen regulatorische Anforderungen, wie der digitale Produktpass (\acs{dpp}), der von der EU entwickelt und künftig für viele Produktgruppen verpflichtend sein wird, die Notwendigkeit eines einheitlichen Konzepts zur standardisierten und semantischen Beschreibung von Produkten.

Um den wachsenden Anforderungen an Transparenz, Rückverfolgb1arkeit und Effizienz gerecht zu werden und technologisch auf dem neuesten Stand zu bleiben, ist es für Unternehmen daher unerlässlich, sich bereits frühzeitig mit der \acs{aas} auseinanderzusetzen.

\subsection{Zielsetzung}
Ziel dieser Arbeit ist die prototypische Modellierung eines digitalen Zwillings für das Abfüll- und Verschließmodul der robocell-Linie der Firma groninger. 
Dabei soll untersucht werden, inwiefern der Einsatz der \acs{aas} dem Unternehmen einen Mehrwert bietet, insbesondere im Hinblick auf Interoperabilität, Produktivität und Nachverfolgbarkeit.

Unter Interoperabilität wird die herstellerunabhängige und standardisierte Kommunikation zwischen Maschinen und Systemen verstanden, die eine nahtlose Vernetzung in einem Unternehmen als auch über Unternehmensgrenzen hinweg ermöglicht.
Hinsichtlich der Produktivität steht die effiziente Bereitstellung und der Austausch von Informationen im Fokus, mit Potenzialen zur Optimierung von Prozessen, etwa durch vorausschauende Instandhaltungsmaßnahmen. 
Die Nachverfolgbarkeit schließlich umfasst die lückenlose Erfassung und Rückverfolgung aller relevanten Asset-Daten über den gesamten Lebenszyklus, wodurch Zustandsänderungen und Ereignisse transparent und nachvollziehbar werden.

Darüber hinaus soll das Potenzial von Künstlicher Intelligenz (\acs{ki}) zur Analyse und Nutzung der im digitalen Zwilling erfassten Daten betrachtet werden, um mögliche Optimierungsansätze im Betrieb aufzuzeigen.
Die gewonnenen Erkenntnisse dienen als Grundlage für eine Handlungsempfehlung, die den Nutzen und die Relevanz des Einsatzes der \acs{aas} im konkreten Anwendungskontext bei groninger bewertet.

\subsection{Vorgehensweise}

Die Arbeit beginnt mit einer eingehenden Betrachtung grundlegender Begriffe und Konzepte (Kapitel \ref{sec:Grundlagen}).
Insbesondere wird erläutert, was unter \acs{ki}, einem digitalen Zwilling sowie der \acs{aas} zu verstehen ist und wie sich diese im industriellen Kontext, insbesondere im Rahmen von Industrie 4.0, einordnen lassen. 
Ergänzend werden zentrale Technologien und Werkzeuge vorgestellt, die als methodische Grundlage für die anschließende technische Umsetzung dienen.

Darauf aufbauend erfolgt in Kapitel \ref{sec:Entwicklung} die Modellierung eines digitalen Zwillings für das Abfüll -und Verschließmodul der robocell-Linie auf Basis der \acs{aas}.
Zudem wird ein möglicher Einsatz von \acs{ki} zur datenbasierten Optimierung aufgezeigt und prototypisch realisiert.
Abschließend werden zwei praxisnahe Anwendungsfälle zum \acs{dpp} sowie zur automatisierten Generierung von AAS umgesetzt.

Im Ergebnisteil (Kapitel \ref{sec:Ergebnisse}) wird der entwickelte \acs{aas}-Demonstrator vorgestellt und zentrale Herausforderungen bei der Implementierung analysiert.
Darüber hinaus werden das eingesetzte KI-Modell sowie die beiden Anwendungsfälle präsentiert und bewertet.
Zusätzlich erfolgt eine Evaluation der verwendeten Open-Source-Lösungen zur Modellierung und Bereitstellung der \acs{aas}.

Kapitel \ref{sec:Zusammenfassung} bildet den Abschluss der Arbeit und fasst die zentralen Inhalte und Ergebnisse kompakt zusammen. 
Basierend darauf wird eine konkrete Handlungsempfehlung für das Unternehmen groninger abgeleitet sowie ein Ausblick auf zukünftige Entwicklungen und mögliche Erweiterungen gegeben.

