\thispagestyle{fancy}
\section{Einleitung}
\subsection{Motivation}
\label{sec:Motivation}
Im Zuge der vierten industriellen Revolution gewinnen digitale Zwillinge zunehmend an Bedeutung. 
Sie ermöglichen die virtuelle Abbildung physischer Assets und schaffen damit die Grundlage für transparentere und effizientere industrielle Prozesse. 
Viele Unternehmen setzen bereits auf solche digitalen Repräsentationen, verwenden dabei jedoch häufig proprietäre Lösungen, die nicht interoperabel sind und den Austausch von Daten über System- und Unternehmensgrenzen hinweg erschweren.

Mit der steigenden Nachfrage nach digitalen Zwillingen wächst auch der Bedarf an standardisierten und herstellerunabhängigen Lösungen. 
Als Antwort auf diese Herausforderung hat die Plattform Industrie 4.0 die \ac{aas} etabliert. 
Sie bietet ein einheitliches Rahmenwerk zur interoperablen Umsetzung digitaler Zwillinge und erlaubt es, Informationen zu einem Asset über dessen gesamten Lebenszyklus hinweg digital zu erfassen und strukturiert bereitzustellen.

Die zunehmende Umsetzung der \acs{aas} in Pilotprojekten sowie erste produktive Anwendungen verdeutlichen ihre hohe Relevanz und das Potenzial für die industrielle Praxis. 
Gleichzeitig führen regulatorische Vorgaben wie der von der Europäischen Union (\acs{eu}) geplante Digitale Produktpass (\acs{dpp}), der künftig für zahlreiche Produktgruppen verpflichtend sein wird, die Notwendigkeit eines standardisierten und semantisch eindeutigen Informationsmodells deutlich vor Augen.
Vor diesem Hintergrund gewinnt die Auseinandersetzung mit der \acs{aas} für Unternehmen bereits heute an Bedeutung, um den steigenden Anforderungen an Transparenz, Rückverfolgbarkeit und Effizienz gerecht zu werden.

\subsection{Zielsetzung}

Ziel dieser Arbeit ist es, am Beispiel des Abfüll- und Verschließmoduls der robocell-Linie der Firma groninger das Potenzial der \acs{aas} für die Umsetzung digitaler Zwillinge zu untersuchen. 
Der Fokus liegt auf dem Mehrwert, den die \acs{aas} insbesondere in Bezug auf Interoperabilität, Produktivität und Nachverfolgbarkeit bieten kann. 

Interoperabilität bezeichnet in diesem Zusammenhang die standardisierte und hersteller\-unabhängige Vernetzung von Maschinen und Systemen, die einen konsistenten Datenaustausch innerhalb und zwischen Unternehmen ermöglicht. 
Produktivität umfasst die effiziente Bereitstellung von Informationen, die Optimierungspotenziale für betriebliche Prozesse eröffnet. 
Nachverfolgbarkeit umfasst die lückenlose Erfassung relevanter Asset-Daten über den gesamten Lebenszyklus, wodurch Zustandsänderungen transparent und nachvollziehbar werden.

Darüber hinaus soll das Potenzial von Künstlicher Intelligenz (\acs{ki}) zur Analyse und Nutzung der im digitalen Zwilling erfassten Daten betrachtet werden, um mögliche Optimierungsansätze im Betrieb aufzuzeigen. 
Ebenso soll die Praxistauglichkeit der eingesetzten Open-Source-Lösungen zur Modellierung und Bereitstellung der \acs{aas} untersucht werden.
Die gewonnenen Erkenntnisse dienen als Grundlage für eine Handlungsempfehlung, die Nutzen und Relevanz des Einsatzes der \acs{aas} im konkreten Anwendungskontext bei groninger bewertet.

\subsection{Vorgehensweise}

Die Arbeit beginnt mit einer Darstellung der theoretischen Grundlagen (Kapitel \ref{sec:Grundlagen}). 
Dazu gehören zentrale Konzepte wie \acs{ki}, der digitale Zwilling sowie insbesondere die \acs{aas}, die allesamt eine zentrale Rolle im Rahmen von Industrie 4.0 einnehmen. 
Darüber hinaus werden mit dem AASX Package Explorer, Eclipse BaSyx und \ac{opcua} für die weitere Vorgehensweise relevante Werkzeuge und Technologien eingeführt, die die methodische Basis für die anschließende Entwicklung bilden.

Darauf aufbauend wird in Kapitel \ref{sec:Entwicklung} der digitale Zwilling für das Abfüll- und Verschließmodul der robocell prototypisch implementiert. 
Dies umfasst sowohl die Modellierung der \acs{aas} als auch ihre Integration in ein Industrie-4.0-kompatibles System, das durch die Anbindung dynamischer Datenquellen erweitert wird. 
Außerdem wird ein Verfahren zur Anomalieerkennung mithilfe von \acs{ki} konzipiert und prototypisch umgesetzt. 
Abschließend werden mit dem \acs{dpp} sowie der automatisierten Generierung von \acs{aas} zwei praxisnahe Anwendungsfälle ausgearbeitet.

Im Ergebnisteil (Kapitel \ref{sec:Ergebnisse}) wird der entwickelte \acs{aas}-Demonstrator vorgestellt, das \acs{ki}-Modell evaluiert und die beiden Anwendungsfälle reflektiert. 
Zudem wird eine Bewertung der eingesetzten Werkzeuge hinsichtlich ihrer Praxistauglichkeit vorgenommen.

Kapitel \ref{sec:Zusammenfassung} fasst die zentralen Ergebnisse der Arbeit zusammen, leitet eine Handlungs\-empfehlung für das Unternehmen groninger ab und schließt mit einem Ausblick auf mögliche Weiterentwicklungen.
