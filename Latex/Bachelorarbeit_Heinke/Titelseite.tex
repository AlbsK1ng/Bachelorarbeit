\begin{titlepage}

\begin{singlespace}
    
\begin{tikzpicture}[remember picture, overlay]
     \node [shift={(-17 cm,-2.25cm)}]  at (current page.north east)
         {
            \includegraphics[height=1.4cm]{./Bilder/groninger_Logo_Transparent_RGB.png}
         };
\end{tikzpicture}
 \begin{tikzpicture}[remember picture, overlay]
     \node [shift={(-3.5 cm,-2.5cm)}]  at (current page.north east)
         {
             \includegraphics[height=2.5cm]{./Bilder/HHN.png}
         };
 \end{tikzpicture}




\vspace{1cm}
\centering
\fontsize{30}{32} \selectfont Implementierung eines digitalen Zwillings für die robocell und Evaluierung eines KI gestützten Konzepts zur Optimierung  
 \par
\vspace{1.5cm}
\fontsize{36}{33} \selectfont Bachelorarbeit \par
\vspace{1.5cm}
\fontsize{16}{20} \selectfont Studiengang Elektrotechnik, SPO 03\\
Hochschule Heilbronn, Campus Künzelsau
\vspace{1.5cm}

\fontsize{18}{22} \selectfont von \par
\fontsize{20}{25} \selectfont Albert Heinke\par
\vspace{1.5cm}
\fontsize{18}{22} \selectfont \today \par
\vfill

\fontsize{13}{18} \selectfont
\begin{tabular}{ l  l }
  Bearbeitungszeitraum & \hspace{0.85cm} 4 Monate \\
  Matrikelnummer, Semester & \hspace{0.85cm} 212081, 8. Fachsemester \\
  Unternehmen, Ort & \hspace{0.85cm} groninger \& co. GmbH, Crailsheim \\
  Betrieblicher Betreuer & \hspace{0.85cm} Jan Völkl, M.Sc. \\
  Erstprüfer & \hspace{0.85cm} Prof. Dr.-Ing. Ralf Gessler \\
  Zweitprüfer & \hspace{0.85cm} Prof. Dr.-Ing. Marcus Stolz \\
\end{tabular}

\newpage


%\pagenumbering{gobble}

\pagenumbering{Roman}
\setcounter{page}{2}

\renewcommand{\contentsname}{Inhaltsverzeichnis}

\fontsize{11}{18} \selectfont

\end{singlespace}
\end{titlepage}