\section{Entwicklung}
In diesem Kapitel wird die Entwicklung des digitalen Zwillings für die robocell beschrieben.
\subsection{Konzeptionierung des digitalen Zwilling}


\subsubsection{Identifikation relevanter Datenquellen}
Ein digitaler Zwilling basiert immmer auf einer Vielzahl unterschiedlicher Daten. 
Daher ist es essenziell, im ersten Schritt alle relevanten Datenquellen zu identifizieren.
In industriellen Umgebungen existieren typischerweise verschiedene Systeme zur Erfassung, Verwaltung und Speicherung von Maschinendaten.
Bei groninger übernimmt diese Funktion das \acs{plm}-System Agile, das eng mit dem \acs{erp}-System PSI Penta verknüpft ist.
Darin können unter anderem technische Spezifikation, Stücklisten, \acs{cad}-Dateien oder allgemeine Dokumente gefunden werden, die die Basis des digitalen Zwillings der robocell bilden.

Neben statischen Informationen aus den Unternehmenssystemen spielen aber auch dynamische Daten, wie sie durch Sensoren oder Steuerungssysteme erzeugt werden, eine wichtige Rolle.
Da im Rahmen dieser Arbeit keine reale Maschine angebunden ist, werden diese Daten simuliert.
Hierfür wird eine in node.js entwickelte Anwendung eingesetzt, die sowohl Prozess- als auch Betriebsdaten generiert. 
Ergänzend dazu wird ein Maschinensimulator verwendet, der einen PackML-Zustandsautomaten abbildet und typische Maschinenzustände sowie deren Übergänge simuliert. 
Beide Komponenten stehen als Docker Container zur Verfügung und stellen die Daten über einen \acs{opcua} Server bereit, wodurch eine realitätsnahe Datenbasis geschaffen wird.
\subsubsection{Auswahl geeigneter Teilmodelle}
Aufbauend auf den zuvor betrachteten Informationsquellen gilt es nun zu entscheiden, welche Aspekte der Maschine in der \acs{aas} modelliert werden sollen.
Als Orientierung dienen die von der \acs{idta} bereitgestellten Submodel Templates, die bereits viele typische Anwendungsfälle standardisiert abdecken.
Darüber hinaus können jedoch auch eigene Teilmodelle entwickelt werden, die gezielt auf projektspezifische Anforderungen zugeschnitten sind.
Diese können entweder vollständig neu konzipiert oder aus bestehenden Vorlagen abgeleitet werden.

Nachfolgende Tabelle gibt einen Überblick über die in diesem Projekt eingesetzten Submodelle, ihre jeweilige Funktion, die Datenquell und ob es für diese bereits Vorlagen gibt.



\subsection{Modellierung mit der AAS}
\subsubsection{Umsetzung mit dem AASX Package Explorer}
\subsubsection{Validieren und Testen}
\subsection{Technische Integration}
\subsubsection{Bereitstellung der Verwaltungsschalen}
\subsubsection{Datenzugriff über standardisierte Schnittstellen}
\subsubsection{Integration von Echtzeitdaten über OPC UA}
\subsubsection{Verarbeitung von Zeitreihendaten}
\subsection{Anwendungsfall Digitaler Produktpass}
\subsubsection{Beschreibung}
\subsubsection{Umsetzung mit dem Teilmodell Carbon Footprint}
\subsection{Anwendungsfall automatisierte Generierung von AAS}
\subsubsection{Erstellen von Submodell-Templates}
\subsubsection{Befüllen der Templates mit strukturierten Daten}
\subsubsection{Bereitstellen der AAS über die Rest API}
\subsubsection{Potenziale des KI-Einsatzes}