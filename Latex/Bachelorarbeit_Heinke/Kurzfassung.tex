\newpage
\phantomsection
\pdfbookmark[1]{Kurzfassung}{kurzfassung}
\section*{Kurzfassung}

% EInleitung/Ziel Präsens
Die fortschreitende Digitalisierung im Rahmen von Industrie 4.0 führt zu einem wachsenden Bedarf an standardisierten Konzepten zur digitalen Repräsentation physischer Assets. 
Die \ac{aas} stellt hierfür einen vielversprechenden Ansatz dar, da sie die Grundlage für die Umsetzung interoperabler digitaler Zwillinge bildet. 
Ziel dieser Arbeit ist es, den praktischen Mehrwert sowie die prototypische Implementierung dieser Lösung zu untersuchen. 
Als Anwendungsbeispiel dient das Abfüll- und Verschließmodul der robocell-Linie von groninger.\\
FÜr dieses wurde ein digitaler Zwilling in Form eines \acs{aas}-Demonstrators entwickelt.
Die Umsetzung umfasst die statische Modellierung von Stammdaten sowie die Anbindung und Integration simulierter Echtzeitdaten über ein Industrie-4.0.kompatibles System.
Ergänzend wurden zwei Anwendungsfälle umgesetzt. 
Einer befasste sich mit der automatisierten Generierung von AAS-Instanzen zur Bewertung der Praxistauglichkeit in industriellen Anwendungsszenarien, der andere mit der Integration eines \ac{dpp}, um die Eignung der \acs{aas} für zukünftige regulatorische Anforderungen zu untersuchen.
Nicht zuletzt wurde ein KI-Modell zur Anomalieerkennung konzipiert und exemplarisch umgesetzt, um das Potenzial datengetriebener Optimierung aufzuzeigen.\\
Die Ergebnisse der Arbeit zeigen, das mit der AAS bereits heute grundsätzlich interoperable, maschinenlesbare und nachvollziehabre DT umsetzen lassen, auch bei groninger.
ABer auch das für einenn produktiven Einsatz mehr dazu gehört als lediglich eine rein manuelle Modellierung, wie mit der automatisierten Generierung demonstriert.
Darüber zeigt es das für hinaus für DPP geeignet.
KI ich wies nicht ob ich das shcon zuvor hatte aber das wäre ja hier angebracjt die ERgebnisse der Anomalierkennung zeigenm, das mit einem Autoencoder Abweichugnen in Zwitreihendaten erkannt werden könen, die dazu genutzt werden könnten die zustandsbasierte Überwachhung zu ooptimieren.

%
Die fortschreitende Digitalisierung im Rahmen von Industrie 4.0 führt zu einem wachsenden Bedarf an standardisierten Konzepten zur digitalen Repräsentation physischer Assets. Die Asset Administration Shell (AAS) stellt hierfür einen vielversprechenden Ansatz dar, da sie die Grundlage für die Umsetzung interoperabler digitaler Zwillinge bildet. Ziel dieser Arbeit ist es, den praktischen Mehrwert sowie die prototypische Implementierung dieser Lösung zu untersuchen. Als Anwendungsbeispiel dient das Abfüll- und Verschließmodul der robocell-Linie von groninger. Für dieses wurde ein digitaler Zwilling in Form eines AAS-Demonstrators entwickelt. Die Umsetzung umfasst die statische Modellierung von Stammdaten sowie die Anbindung und Integration simulierter Echtzeitdaten über ein Industrie-4.0-kompatibles System. Ergänzend wurden zwei Anwendungsfälle realisiert: Einer befasst sich mit der automatisierten Generierung von AAS-Instanzen zur Bewertung der Praxistauglichkeit in industriellen Anwendungsszenarien, der andere mit der Integration eines digitalen Produktpasses (DPP), um die Eignung der AAS für zukünftige regulatorische Anforderungen zu untersuchen. Nicht zuletzt wurde ein KI-Modell zur Anomalieerkennung konzipiert und exemplarisch umgesetzt, um das Potenzial datengetriebener Optimierung aufzuzeigen. Die Ergebnisse der Arbeit zeigen, dass sich mit der AAS bereits heute grundsätzlich interoperable, maschinenlesbare und nachvollziehbare digitale Zwillinge realisieren lassen – auch im Kontext bestehender Anlagen bei groninger. Gleichzeitig wird deutlich, dass für einen produktiven Einsatz mehr erforderlich ist als eine rein manuelle Modellierung, wie die automatisierte Generierung exemplarisch verdeutlicht. Darüber hinaus zeigt die Umsetzung des digitalen Produktpasses, dass die AAS prinzipiell für regulatorische Anforderungen geeignet ist. Die Ergebnisse der Anomalieerkennung belegen zudem, dass mithilfe eines Autoencoders Abweichungen in Zeitreihendaten zuverlässig identifiziert werden können, was eine Grundlage für die Optimierung zustandsbasierter Prozessüberwachung darstellt.
%
