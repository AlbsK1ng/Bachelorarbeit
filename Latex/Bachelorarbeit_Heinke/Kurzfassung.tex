\newpage
\phantomsection
\pdfbookmark[1]{Kurzfassung}{kurzfassung}
\section*{Kurzfassung}

Die fortschreitende Digitalisierung im Rahmen von Industrie 4.0 und die daraus resultierende Vernetzung von Wertschöpfungsketten erfordern standardisierte Konzepte zur digitalen Repräsentation physischer Assets.
% Die fortschreitende Digitalisierung im Rahmen von Industrie 4.0 und die steigende Nachfrage nach Transparenz erfordern standardisierte Konzepte zur digitalen Repräsentation physischer Assets.
Die \ac{aas} stellt hierfür einen vielversprechenden Ansatz dar, da sie die Grundlage für interoperable digitale Zwillinge bildet.
Ziel dieser Arbeit ist es, den praktischen Mehrwert dieser Lösung am Beispiel des Abfüll- und Verschließmoduls der robocell-Linie der Firma groninger zu untersuchen.
Hierzu wurde ein \acs{aas}-Demonstrator entwickelt, der statische Stammdaten mit dynamischen Echtzeit- und Zeitreihendaten verbindet und damit einen voll funktionsfähigen digitalen Zwilling bildet.
Ergänzend wurde ein auf einem Autoencoder basierendes \ac{ki}-Modell prototypisch umgesetzt, das Anomalien in simulierten Prozessdaten identifizierte und damit das Potenzial zur Optimierung durch zustandsbasierte Überwachung aufzeigte.
Darüber hinaus wurden zwei praxisnahe Anwendungsfälle ausgearbeitet. 
Mit dem Digitalen Produktpass (\acs{dpp}) konnte gezeigt werden, dass regulatorische Anforderungen im Kontext von Nachhaltigkeit standardisiert abgebildet und differenziert für verschiedene Interessengruppen bereitgestellt werden können, während die automatisierte Generierung von \ac{aas}-Instanzen demonstrierte, wie sich der manuelle Modellierungsaufwand bei der Erstellung digitaler Zwillinge verringern lässt.
Die Ergebnisse belegen, dass die \acs{aas} bereits heute eine tragfähige Grundlage für digitale Zwillinge bietet, für den industriellen Einsatz jedoch ergänzende Automatisierungsmechanismen, Systemintegrationen und Sicherheitskonzepte erforderlich sind.

% \newpage
% \phantomsection
% \pdfbookmark[1]{Abstract}{abstract}
% \section*{Abstract}
% % Englisch
% The ongoing digitalization within the framework of Industry 4.0 and the resulting interconnection of value chains require standardized concepts for the digital representation of physical assets. The Asset Administration Shell (AAS) represents a promising approach, as it provides the foundation for interoperable digital twins. The aim of this thesis is to investigate the practical value of this solution using the example of the filling and closing module of the robocell production line by groninger. For this purpose, an AAS demonstrator was developed that integrates static master data with dynamic real-time and time-series data, thereby creating a fully functional digital twin. In addition, a prototype AI model based on an autoencoder was implemented, which identified anomalies in simulated process data and thus demonstrated the potential of condition-based monitoring. Furthermore, two practical use cases were elaborated. The Digital Product Passport (DPP) showed that regulatory requirements in the context of sustainability can be represented in a standardized manner and differentiated for various stakeholders, while the automated generation of AAS instances demonstrated how the manual modeling effort in creating digital twins can be reduced. The results confirm that the AAS already provides a viable foundation for digital twins today, but that additional automation mechanisms, system integrations, and security concepts are necessary for industrial deployment.