\documentclass[12pt]{extarticle}
\usepackage[utf8]{inputenc}
\usepackage{amsmath}

% Times Roman
\renewcommand{\rmdefault}{ptm}
% \usepackage{lmodern} Computer Modern
% \usepackage[square,numbers]{natbib}
\usepackage{graphicx}
\usepackage{tikz}
\usepackage{pdfpages}
\usepackage[ngerman]{babel}
\usepackage{caption}
\usepackage{subcaption}
\usepackage{tocbibind}
\usepackage{hyperref}
% \usepackage{cleveref}
% cleveref sorgt für Probleme mit acronym

%\usepackage{geometry}
\usepackage[onehalfspacing]{setspace}
\usepackage{color}
\usepackage{attachfile}
\usepackage[explicit]{titlesec}
\usepackage{lipsum}
\usepackage{blindtext,listings}
% \usepackage{float}
% \usepackage{textcomp}
\usepackage{microtype} %gegen flattersatz
\usepackage[printonlyused]{acronym} %Abkürzungen
\usepackage{float}
\usepackage{booktabs}
\usepackage{eurosym}
\usepackage{glossaries}
\usepackage{fancyhdr} %Header Paket
\usepackage[toc,page]{appendix}


% Hinweis Box
\usepackage[T1]{fontenc} 
\usepackage{blindtext} 
\usepackage{xcolor} 
\usepackage{framed} 

% Automatische Absaätze
\usepackage{parskip}

% % Literatur
\usepackage[backend=bibtex,style=numeric-comp,sorting=none]{biblatex}
\addbibresource{references.bib}

% Hinweis-Box
\usepackage{tcolorbox}
%Tabellen
\usepackage{booktabs}
\usepackage{multirow}
\usepackage{ltablex}
\usepackage{rotating}
\usepackage{tabularray}
\usepackage{makecell}
\usepackage{adjustbox} 
%----------Code listings------------
\usepackage{listings}
\usepackage[scaled=1]{inconsolata}
% Farben definieren
\definecolor{numb}{rgb}{0.0, 0.5, 0.0}
\definecolor{string}{rgb}{0.4, 0.0, 0.4}
\definecolor{background}{HTML}{F5F5F5}
\definecolor{darkgray}{gray}{0.1}
\lstdefinelanguage{json}{
    numbers=left,
    numberstyle=\small,
    backgroundcolor=\color{background},
    numbersep=8pt,
    frame=single,
    rulecolor=\color{black},
    showspaces=false,
    showtabs=false,
    breaklines=true,
    postbreak=\raisebox{0ex}[0ex][0ex]{\ensuremath{\color{gray}\hookrightarrow\space}},
    breakatwhitespace=true,
    basicstyle=\ttfamily\small,
    lineskip=-0.25pt, 
    upquote=true,
    morestring=[b]",
    stringstyle=\color{black},
    literate=
     *{0}{{{\color{numb}0}}}{1}
      {1}{{{\color{numb}1}}}{1}
      {2}{{{\color{numb}2}}}{1}
      {3}{{{\color{numb}3}}}{1}
      {4}{{{\color{numb}4}}}{1}
      {5}{{{\color{numb}5}}}{1}
      {6}{{{\color{numb}6}}}{1}
      {7}{{{\color{numb}7}}}{1}
      {8}{{{\color{numb}8}}}{1}
      {9}{{{\color{numb}9}}}{1}
      {\{}{{{\color{black}{\{}}}}{1}
      {\}}{{{\color{black}{\}}}}}{1}
      {[}{{{\color{black}{[}}}}{1}
      {]}{{{\color{black}{]}}}}{1},
}
%%% inline %%%
\tcbset{inlinebox/.style={
  on line,
  box align=base,
  colback=gray!20,
  colframe=gray!50,
  boxrule=0.5pt,
  arc=2pt,
  top=1pt,
  bottom=1pt,
  left=2pt,
  right=2pt
}}
%%% inline %%%
%----------Code listings------------

%----------Autoencoder-------------------
\usepackage{neuralnetwork}
\newcommand{\xin}[2]{$x_#2$}
\newcommand{\xout}[2]{$\hat x_#2$}
%----------Autoencoder-------------------

%----------Farben-------------------
\definecolor{swaggerget}{HTML}{61affe}
\definecolor{swaggerpost}{HTML}{49cc90}
\definecolor{swaggerput}{HTML}{fca130}
\definecolor{swaggerdelete}{HTML}{f93e3e}
\definecolor{babyblue}{HTML}{F2F5FD}
\definecolor{DynamischesSubmodel}{HTML}{dbe2d7}
% \definecolor{tableHeader}{HTML}{d8e4fc}
\definecolor{tableHeader}{RGB}{230,230,230}
\definecolor{rowHeader}{RGB}{245,245,245} 
%----------Farben-------------------

%----------Aufzählung-------------------
\usepackage{enumitem}
\usepackage{pifont}
\usepackage{amssymb}
\definecolor{orangeHaken}{HTML}{aa7e5e}
\definecolor{greenHaken}{HTML}{49cc90}

\newcommand{\recommended}{\textcolor{orangeHaken}{\ding{52}}} % orange Kreis
\newcommand{\optional}{\textcolor{black}{\ding{118}}}
\newcommand{\notused}{\textcolor{black}{\ding{55}}}     % X-Symbol
\newcommand{\mandatory}{\textcolor{greenHaken}{\ding{52}}} % grüner Haken

%%% legende %%%
\tcbset{legendbox/.style={
  colback=background,
  colframe=white,
  arc=2mm,
  boxrule=0pt,
  left=2mm,
  right=2mm,
  top=1mm,
  bottom=1mm,
  boxsep=4pt,
  width=\textwidth
}}
%----------Aufzählung-------------------
\usepackage[a4paper, asymmetric, left=3cm,right=3cm,top=1cm,bottom=1.5cm,includeheadfoot, headheight=40pt]{geometry}

\pagestyle{plain} %nur Seitenzahl in der Fußzeile (LaTeX-Standard)


\renewcommand*{\listoffigures}{%
  \begingroup
  \tocchapter
  \tocfile{\listfigurename}{lof}
  \endgroup
  
  
}

% Befehl zum Schreiben von Quellenangeben in Bildern
\newcommand*{\quelle}{%
    \vspace{2mm}
    \footnotesize Quelle:
}


\loadglsentries{glossary}


\makeglossaries

% 4. Unterpunkt
\setcounter{secnumdepth}{4}
\setcounter{tocdepth}{4}

\begin{document}

\begin{titlepage}

\begin{singlespace}
    
\begin{tikzpicture}[remember picture, overlay]
     \node [shift={(-17 cm,-2.25cm)}]  at (current page.north east)
         {
            \includegraphics[height=1.4cm]{./Bilder/groninger_Logo_Transparent_RGB.png}
         };
\end{tikzpicture}
 \begin{tikzpicture}[remember picture, overlay]
     \node [shift={(-3.5 cm,-2.5cm)}]  at (current page.north east)
         {
             \includegraphics[height=2.5cm]{./Bilder/HHN.png}
         };
 \end{tikzpicture}




\vspace{1cm}
\centering
\fontsize{30}{32} \selectfont Implementierung eines digitalen Zwillings für die robocell und Evaluierung eines KI gestützten Konzepts zur Optimierung  
 \par
\vspace{1.5cm}
\fontsize{36}{33} \selectfont Bachelorarbeit \par
\vspace{1.5cm}
\fontsize{16}{20} \selectfont Studiengang Elektrotechnik, SPO 03\\
Hochschule Heilbronn, Campus Künzelsau
\vspace{1.5cm}

\fontsize{18}{22} \selectfont von \par
\fontsize{20}{25} \selectfont Albert Heinke\par
\vspace{1.5cm}
\fontsize{18}{22} \selectfont \today \par
\vfill

\fontsize{13}{18} \selectfont
\begin{tabular}{ l  l }
  Bearbeitungszeitraum & \hspace{0.85cm} 4 Monate \\
  Matrikelnummer, Semester & \hspace{0.85cm} 212081, 8. Fachsemester \\
  Unternehmen, Ort & \hspace{0.85cm} groninger \& co. GmbH, Crailsheim \\
  Betrieblicher Betreuer & \hspace{0.85cm} Jan Völkl, M.Sc. \\
  Erstprüfer & \hspace{0.85cm} Prof. Dr.-Ing. Ralf Gessler \\
  Zweitprüfer & \hspace{0.85cm} Prof. Dr.-Ing. Marcus Stolz \\
\end{tabular}

\newpage


%\pagenumbering{gobble}

\pagenumbering{Roman}
\setcounter{page}{2}

\renewcommand{\contentsname}{Inhaltsverzeichnis}

\fontsize{11}{18} \selectfont

\end{singlespace}
\end{titlepage}
% pdf zur Tätigkeitsübersicht
% pdf zur studentischer Reflexion


\newpage
\setcounter{page}{2}

%\includepdf{Deckblatt/sperrvermerk.pdf} 

%\includepdf{Deckblatt/ehrenwoertlErklaerung.pdf} 
% nach Unterschreiben der von Sperrvermerk und ehrenwörtlicher Erklärung, hier als eingescanntes PDF einfügen und Latexteil auskommentieren

% \subsection*{Sperrvermerk}
% Die vorliegende Arbeit beinhaltet interne vertrauliche Informationen des Unternehmens groninger \& Co. GmbH. Sie ist nur für die Beteiligten am Begutachtungs- und Evaluationsprozess bestimmt. Die Weitergabe des Inhalts der Arbeit im Gesamten oder in Teilen sowie das Anfertigen von Kopien oder Abschriften – auch in digitaler Form – sind grundsätzlich untersagt. Ausnahmen bedürfen der schriftlichen Genehmigung des Unternehmens groninger \& Co. GmbH. \\ \\ \\ \\

% \hspace{2cm}
% \parbox{9cm}{\centering M.Sc. Völkl, Jan \hrule
% \strut \centering\footnotesize Name, Vorname des Betreuers/Gutachters/Prüfers \\ (Bitte in Druckbuchstaben)} \\ \\ \\

% \parbox[t][][t]{5cm}{\centering Crailsheim, \today \hrule \strut \centering \footnotesize Ort, Datum} %\hfill
% \hspace{3cm}
% \parbox[t][][t]{5cm}{\vspace{0.09cm} \hrule \strut \centering \footnotesize  Unterschrift des Betreuer/Gutachter/Prüfer}


\thispagestyle{empty}
\begin{center}
\vspace*{5cm}
\textbf{\Huge HINWEIS}

\vspace{1cm}
\textbf{\Large Die Seite Sperrvermerk im fertigen Dokument als signierte PDF einfügen.}

\end{center}
\newpage

\subsection*{Ehrenwörtliche Erklärung}

Ich erkläre hiermit ehrenwörtlich, dass ich die vorliegende Arbeit selbständig angefertigt habe; die aus fremden Quellen direkt oder indirekt übernommenen Gedanken sind als solche kenntlich gemacht.\\ \\ 
Die Arbeit wurde bisher keiner anderen Prüfungsbehörde vorgelegt und auch noch nicht veröffentlicht.
\\ \\ \\
\parbox[t][][t]{5.5cm}{\centering Crailsheim, 01. September 2025 \hrule
\strut \centering\footnotesize Ort, Datum} \hfill
\parbox[t][][t]{5cm}{\hrule\strut \centering\footnotesize Studierende/r}



\newpage
\begin{spacing}{1.0}
\tableofcontents %Inhaltsverzeichnis
\newpage
\end{spacing}

\begin{spacing}{1.0}
  \section*{Abkürzungsverzeichnis}
\begin{singlespacing}
\end{singlespacing}
\begin{acronym}

\acro{aas}[AAS]{Asset Administration Shell}
\acro{idta}[IDTA]{Industrial Digital Twin Association}
\acro{dpp}[DPP]{Digitaler Produktpass}
\acroplural{dpp}[DPPs]{Digitale Produktpässe}

\acro{opcua}[OPC UA]{Open Platform Communications Unified Architecture}

\acro{rami}[RAMI 4.0]{Refernzarchitekturmodell Industrie 4.0}

\acro{ki}[KI]{Künstliche Intelligenz}

\end{acronym}



\newpage
  \newpage
  \end{spacing}

\listoffigures
\newpage

\listoftables
\newpage

\lstlistoflistings 

\newpage
\thispagestyle{empty}
\cleardoublepage

\pagenumbering{arabic}
%arabische Seitenzahlen im Hauptteil

\pagestyle{fancy}
\fancyhf{}
\renewcommand{\headrulewidth}{0.4pt} %obere Trennlinie
\renewcommand{\sectionmark}[1]{\markright{\arabic{section}.\ #1}}
\renewcommand{\subsectionmark}[1]{}

\renewcommand{\footrulewidth}{0.4pt}% default is 0pt

\fancyhead[LE, RO]{\rightmark}

\fancyfoot[LE, RO]{\thepage}% Seitenzahl in Fußzeile (Links auf ungeraden LO und Rechts auf geraden Seiten RE) 
%\fancyfoot[LE, RO]{Albert Heinke 2025}% Name und Jahreszahl in Fußzeile (Links auf geraden LE und Rechts auf ungeraden Seiten RO)

\input{commands.tex}
%=======================BEGIN HAUPTTEIL===================================%
\thispagestyle{fancy}
\section{Einleitung}
\subsection{Motivation}
\label{sec:Motivation}
Im Zuge der vierten industriellen Revolution gewinnen digitale Zwillinge zunehmend an Bedeutung. 
Sie ermöglichen die digitale Abbildung physischer Assets und schaffen damit die Grundlage für transparentere und effizientere industrielle Prozesse. 
Viele Unternehmen setzen bereits auf solche digitalen Abbilder, verwenden dabei jedoch häufig proprietäre Lösungen, die nicht interoperabel sind und den Austausch von Daten über System- und Unternehmensgrenzen hinweg erschweren.

Mit der steigenden Nachfrage nach digitalen Zwillingen wächst der Bedarf an standardisierten und herstellerunabhängigen Lösungen. 
Als Antwort auf diese Herausforderung hat die Plattform Industrie 4.0 die sogenannte \ac{aas} etabliert. 
Sie bietet ein einheitliches Rahmenwerk zur interoperablen Umsetzung digitaler Zwillinge und erlaubt es, Informationen zu einem Asset über dessen gesamten Lebenszyklus hinweg digital zu erfassen und strukturiert bereitzustellen.

Die zunehmende Umsetzung der \acs{aas} in Pilotprojekten sowie erste produktive Anwendungen verdeutlichen ihre hohe Relevanz und das Potenzial für die industrielle Praxis. 
Gleichzeitig führen regulatorische Vorgaben wie der von der EU geplante digitale Produktpass (\acs{dpp}), der künftig für zahlreiche Produktgruppen verpflichtend sein wird, die Notwendigkeit eines standardisierten und semantisch eindeutigen Informationsmodells deutlich vor Augen.
Vor diesem Hintergrund gewinnt die Auseinandersetzung mit der \acs{aas} für Unternehmen bereits heute an Bedeutung, um den steigenden Anforderungen an Transparenz, Rückverfolgbarkeit und Effizienz gerecht zu werden.

\subsection{Zielsetzung}

Ziel dieser Arbeit ist es, am Beispiel des Abfüll- und Verschließmoduls der robocell-Linie der Firma groninger das Potenzial der \acs{aas} für die Umsetzung digitaler Zwillinge exemplarisch zu untersuchen. 
Der Fokus liegt dabei auf dem Mehrwert, den die \acs{aas} insbesondere in Bezug auf Interoperabilität, Produktivität und Nachverfolgbarkeit bieten kann. 

Interoperabilität bezeichnet in diesem Zusammenhang die standardisierte und herstellerunabhängige Vernetzung von Maschinen und Systemen, die einen konsistenten Datenaustausch innerhalb und zwischen Unternehmen ermöglicht. 
Produktivität umfasst die effiziente Bereitstellung von Informationen, die Optimierungspotenziale für betriebliche Prozesse eröffnet. 
Nachverfolgbarkeit umfasst die lückenlose Erfassung relevanter Asset-Daten über den gesamten Lebenszyklus, wodurch Zustandsänderungen transparent und nachvollziehbar werden.

Darüber hinaus soll das Potenzial von Künstlicher Intelligenz (\acs{ki}) zur Analyse und Nutzung der im digitalen Zwilling erfassten Daten betrachtet werden, um mögliche Optimierungsansätze im Betrieb aufzuzeigen. 
Ebenso soll die Praxistauglichkeit der eingesetzten Open-Source-Lösungen zur Modellierung und Bereitstellung der \acs{aas} untersucht werden.
Die gewonnenen Erkenntnisse dienen als Grundlage für eine Handlungsempfehlung, die Nutzen und Relevanz des Einsatzes der \acs{aas} im konkreten Anwendungskontext bei groninger bewertet.

\subsection{Vorgehensweise}

Die Arbeit beginnt mit einer Darstellung der theoretischen Grundlagen (Kapitel \ref{sec:Grundlagen}). 
Dazu gehören zentrale Konzepte wie \acs{ki}, der digitale Zwilling sowie insbesondere die \acs{aas}, die allesamt eine zentrale Rolle im Rahmen von Industrie 4.0 einnehmen. 
Darüber hinaus werden mit dem Package Explorer, Eclipse BaSyx und OPC~UA für die weitere Vorgehensweise relevante Werkzeuge und Technologien eingeführt, die die methodische Basis für die anschließende Entwicklung bilden.

Darauf aufbauend wird in Kapitel \ref{sec:Entwicklung} der digitale Zwilling für das Abfüll- und Verschließmodul der robocell prototypisch implementiert. 
Dies umfasst sowohl die Modellierung der \acs{aas} als auch ihre Integration in ein Industrie-4.0-kompatibles System, das durch die Anbindung dynamischer Datenquellen erweitert wird. 
Außerdem wird ein Verfahren zur Anomalieerkennung mithilfe von \acs{ki} konzipiert und prototypisch implementiert. 
Abschließend werden mit dem \acs{dpp} sowie der automatisierten Generierung von \acs{aas} zwei praxisnahe Anwendungsfälle ausgearbeitet.

Im Ergebnisteil (Kapitel \ref{sec:Ergebnisse}) wird der entwickelte \acs{aas}-Demonstrator vorgestellt, das \acs{ki}-Modell evaluiert und die beiden Anwendungsfälle reflektiert. 
Zudem wird eine Bewertung der eingesetzten Werkzeuge hinsichtlich ihrer Praxistauglichkeit vorgenommen.

Kapitel \ref{sec:Zusammenfassung} fasst die zentralen Ergebnisse der Arbeit zusammen, leitet eine Handlungsempfehlung für das Unternehmen groninger ab und schließt mit einem Ausblick auf mögliche Weiterentwicklungen.

\newpage
\section{Grundlagen}
\label{sec:Grundlagen}
Dieses Kapitel behandelt die theoretischen und technologischen Grundlagen, die für das Verständnis und die Durchführung dieser Arbeit relevant sind.
Es beginnt mit einer Einführung in die Konzepte von Industrie 4.0, dem digitalen Zwilling und der \acs{ki}.
Anschließend werden die Struktur und Funktion der \acs{aas} sowie die Rolle des \acs{dpp} dargestellt.
Zum Abschluss folgen die technologischen Voraussetzungen für die praktische Realisierung, darunter der AASX Package Explorer, die Open-Source-Plattform Eclipse BaSyx und der Kommunikationsstandard \acs{opcua}.

\subsection{Industrie 4.0}

Der Begriff Industrie 4.0 wurde erstmals im Jahr 2011 im Rahmen eines von der deutschen Bundesregierung initiierten Zukunftsprojekts eingeführt, das auf die Förderung der Informatisierung in der industriellen Fertigung abzielt.
Angestrebt wird eine Stärkung der Wettbewerbsfähigkeit der deutschen Industrie sowie eine Verbesserung der Marktposition deutscher Unternehmen im globalen Wettbewerb.

Industrie 4.0 steht dabei für die vierte industrielle Revolution und beschreibt die umfassende digitale Transformation industrieller Wertschöpfungsprozesse. 
Im Zentrum steht die intelligente Vernetzung von Menschen, Maschinen und Produkten über moderne digitale Kommunikationsnetzwerke, durch die eine weitreichende Integration der physischen mit der digitalen Welt ermöglicht wird.

Zur besseren Einordnung von Industrie 4.0 ist ein Blick auf die vorangegangenen industriellen Revolutionen hilfreich.
Die Industrialisierung begann bereits Mitte des 18. Jahrhunderts in Großbritannien und breitete sich von dort weltweit aus. 
Mit der Entwicklung der ersten Dampfmaschine setzte die erste industrielle Revolution ein. 
Sie ermöglichte erstmals die Mechanisierung der Fertigung durch den Einsatz von Arbeits- und Kraftmaschinen. 
Dadurch konnten manuelle Tätigkeiten zunehmend durch Maschinenkraft ersetzt werden, insbesondere in der Textil-, Eisen- und Stahlindustrie, die zu den ersten Branchen gehörten, die von dieser Entwicklung profitierten.

Die zweite industrielle Revolution setzte gegen Ende des 19. Jahrhunderts ein und war maßgeblich durch den flächendeckenden Einsatz von Elektrizität geprägt. 
Mit der Erfindung elektrischer Antriebe und des Verbrennungsmotors konnten Maschinen nun auch dezentral betrieben werden. Sie waren nicht länger auf zentrale Kraftquellen wie Dampfmaschinen angewiesen. 
Dies ermöglichte eine flexiblere Gestaltung von Produktionsstätten und führte zur Entwicklung einer arbeitsteiligen Massenproduktion mithilfe von Fließ- und Förderbändern.

Ausgehend von dem deutschen Wirtschaftswunder in den sechziger Jahren des 20. Jahrhunderts entstand in den folgenden Jahrzehnten die dritte industrielle Revolution.
Diese zeichnet sich vor allem durch den Einsatz elektronischer Systeme sowie der Informations- und Kommunikationstechnologie zur Automatisierung aus und ist noch bis heute wirksam.

Aufbauend auf den vorangegangenen industriellen Revolutionen strebt die vierte industrielle Revolution eine tiefgreifende Transformation industrieller Produktionsprozesse an. 
Im Fokus steht die Vernetzung von Systemen, die über moderne Internettechnologien miteinander kommunizieren.
Ziel dieser Entwicklung ist es, die industrielle Wertschöpfung deutlich flexibler und effizienter zu gestalten sowie eine stärkere Individualisierung von Produkten zu ermöglichen. 
\cite{Industrie4.0ProduktionAutomatisierung}\cite{EinführungundUmsetzungI4.0}

Trotz der weiten Verbreitung des Begriffs Industrie 4.0 mangelt es in der Literatur und Forschung an einer einheitlichen Definition.
Vor diesem Hintergrund nimmt insbesondere die Plattform Industrie 4.0 \cite{plattform_i40} eine zentrale Rolle in Deutschland ein. 
Dabei handelt es sich um eine gemeinsame Initiative des Bundesministeriums für Wirtschaft und Energie, des Bundesministeriums für Forschung, Technologie und Raumfahrt sowie führender Industrieverbände, Unternehmen, Forschungseinrichtungen und Gewerkschaften, darunter etwa der Verband Deutscher Maschinen- und Anlagenbau, der Bundesverband Informationswirtschaft, Telekommunikation und neue Medien und der \ac{zvei}.

Die Plattform ist maßgeblich an der inhaltlichen und strategischen Ausarbeitung der Industrie 4.0 beteiligt und leistet einen entscheidenden Beitrag zur Begriffsdefinition. 
Sie definiert Industrie 4.0 als
\glqq die vierte industrielle Revolution, einer neuen Stufe der Organisation und Steuerung der gesamten Wertschöpfungskette über den Lebenszyklus von Produkten.
Dieser Zyklus orientiert sich an den zunehmend individualisierten Kundenwünschen und erstreckt sich von der Idee, dem Auftrag über die Entwicklung und Fertigung, die Auslieferung eines Produkts an den Endkunden bis hin zum Recycling, einschließlich der damit verbundenen Dienstleistungen.
Basis ist die Verfügbarkeit aller relevanten Informationen in Echtzeit durch Vernetzung aller an der Wertschöpfung beteiligten Instanzen sowie die Fähigkeit aus den Daten den zu jedem Zeitpunkt optimalen Wertschöpfungsfluss abzuleiten. 
Durch die Verbindung von Menschen, Objekten und Systemen entstehen dynamische, echtzeitoptimierte und selbst organisierende, unternehmensübergreifende Wertschöpfungsnetzwerke, die sich nach unterschiedlichen Kriterien wie beispielsweise Kosten, Verfügbarkeit und Ressourcenverbrauch optimieren lassen \grqq~\cite[S. 8]{plattform_i40_definition}.

\newpage
Die Definition der Plattform Industrie 4.0 verdeutlicht: Industrie 4.0 steht für einen grundlegenden Wandel in der industriellen Wertschöpfung.
Im Zentrum stehen dabei nicht nur neue technologische Möglichkeiten, sondern vor allem das Potenzial, Produktions- und Geschäftsprozesse flexibler, effizienter und nachhaltiger zu gestalten.
Durch die intelligente Vernetzung und die Nutzung von Echtzeitdaten entstehen selbstorganisierende Systeme, die auch über Unternehmensgrenzen hinweg kooperieren und tiefgreifende organisatorische Veränderungen erfordern.
Auf diese Weise werden nicht nur Qualität und Transparenz gesteigert, sondern auch Ressourcen geschont und Kosten reduziert.
Insgesamt trägt Industrie 4.0 somit nicht nur zur Verbesserung der Wirtschaftlichkeit bei, sondern leistet auch einen wichtigen Beitrag zur ökologischen Nachhaltigkeit.

\subsection{Künstliche Intelligenz}
% Künstliche Intelligenz ist eine Schlüsselkomponente der Industrie 4.0.
Mit der zunehmenden Vernetzung von Maschinen und Anlagen entstehen immer größere Datenmengen entlang von Produktionsprozessen.
Diese Daten bieten enorme Potenziale zur Optimierung industrieller Abläufe, beispielsweise zur Steigerung der Anlagenverfügbarkeit, zur Reduzierung von Ausschuss oder zur Verbesserung der Produktqualität.
Zudem erlauben intelligente Produkte, dass Hersteller auch während der Nutzungsphase Informationen über den Einsatz und Zustand ihrer Produkte erhalten.
Auf dieser Grundlage lassen sich neue Dienstleistungen und Anwendungen entwickeln, wie etwa die vorausschauende Instandhaltung oder die dynamische Optimierung von Betriebsparametern.

Damit solche datengetriebenen Ansätze erfolgreich umgesetzt werden können, benötigt es leistungsfähige Technologien zur Analyse und Interpretation.
Genau hier setzt \acs{ki} an.
Insbesondere Methoden des maschinellen Lernens erlauben es, Zusammenhänge in großen Datenmengen zu erkennen, Zustände zu klassifizieren und sogar Vorhersagen über zukünftige Systementwicklungen zu treffen. \cite{KIEinführung} 
\acs{ki} nimmt somit eine zentrale Rolle innerhalb der Industrie 4.0 ein, indem sie einen echten Mehrwert aus den gesammelten Daten generiert.

Allgemein bezeichnet der Begriff \acs{ki} dabei eine Vielzahl von Methoden und Technologien, die es Computern ermöglichen, Aufgaben zu bewältigen, für die normalerweise menschliche Intelligenz erforderlich ist \cite{KIDefinition1}.
Dazu zählen insbesondere die Fähigkeit, aus Erfahrungen (Daten) zu lernen, Problemlösungen zu entwickeln sowie selbstständig Entscheidungen zu treffen \cite{KIDefinition2}.

Ein zentrales Teilgebiet der \acs{ki} ist das maschinelle Lernen.
Es umfasst Algorithmen, die selbstständig aus Daten lernen und darauf basierend Vorhersagen treffen können.
Im Gegensatz zu klassischen Computerprogrammen folgt das Programm keinem Lösungsweg. 
Stattdessen erkennt es Muster in den vorliegenden Daten und kann auf Grundlage dieser bestimmte Aufgaben ausführen \cite{MLDefinition}.
Aufgrund seiner zentralen Bedeutung für zahlreiche Anwendungen gilt maschinelles Lernen häufig auch als Schlüsseltechnologie der \acs{ki} \cite{MLSchlüsseltechnologie}. 

Grundsätzlich wird zwischen überwachtem Lernen (Supervised Learning) und unüberwachtem Lernen (Unsupervised Learning) unterschieden.
Beim überwachten Lernen wird ein Modell anhand eines Datensatzes trainiert, der sowohl Eingabewerte als auch die dazugehörigen Ziel- bzw. Ausgabewerte enthält.
Der Algorithmus lernt, Zusammenhänge und Muster in den Daten zu erkennen, um daraus eine Funktion abzuleiten, die neue Eingaben korrekt vorhersagen kann.
Typische Anwendungsbereiche sind die Klassifikation (z. B. die Einteilung von Bildern in Kategorien) und die Regression (z. B. die Vorhersage numerischer Werte wie Temperatur oder Energieverbrauch).

Beim unüberwachten Lernen hingegen wird ein Modell mit ungelabelten Daten trainiert, das heißt ohne vorgegebene Zielwerte. 
Der Algorithmus versucht selbstständig Strukturen oder Muster in den vorgegebenen Daten zu finden.
Ein häufig eingesetztes Verfahren ist die Clusteranalyse, bei der ähnliche Datenpunkte zu Gruppen, sogenannten Clustern, zusammengefasst werden. 
Eine typische Anwendung ist beispielsweise die Anomalieerkennung, bei der Abweichungen von einem erwarteten Normalverhalten erkannt werden können. \cite{DLDefinition}

Ein besonders leistungsfähiger Ansatz innerhalb des maschinellen Lernens ist das Deep Learning \cite{DLEinordnung}.
Es basiert auf künstlichen neuronalen Netzen, die in ihrer Struktur an das Verhalten von Neuronen im menschlichen Gehirn angelehnt sind.
Während klassische maschinelle Lernmodelle meist aus nur wenigen Schichten bestehen, kommen beim Deep Learning sogenannte tiefe neuronale Netze mit vielen hintereinandergeschalteten Schichten zum Einsatz.
Diese ermöglichen das Erkennen hochkomplexer und abstrakter Zusammenhänge in großen Datenmengen, etwa bei der Bilderkennung oder Sprachverarbeitung. \cite{DLDefinition}

In jüngster Zeit hat sich insbesondere die generative \acs{ki} als Anwendungsfeld etabliert. 
Sie nutzt tiefe neuronale Netze, um neue Inhalte wie Text, Bilder oder Sprache zu erzeugen. 
Dabei kommen verschiedene Methoden des Deep Learnings zum Einsatz, die aus großen Datenmengen lernen, wie neue Inhalte generiert werden können \cite{GenerativeKI}.

\newpage
\subsection{Digitaler Zwilling}
\label{sec: DT}
Digitale Zwillinge gelten als eine der Schlüsseltechnologien der Industrie 4.0.
Als digitales Gegenstück eines physischen Objekts, sei es eine Maschine, ein Produkt oder eine komplette Anlage, bilden sie dessen Zustand, Verhalten und Leistung virtuell ab.
Dadurch ermöglichen sie eine konsistente Erfassung von Daten, das Simulieren von Prozessen und das frühzeitige Erkennen von Optimierungspotenzialen.

Das Konzept selbst wurde erstmals von Michael Grieves im Jahr 2003 in einer Präsentation zum \ac{plm} vorgestellt. 
Grieves definierte drei grundlegende Komponenten \cite{DTGrieves}, die zusammen das Informationsmodell des digitalen Zwillings bilden:
\begin{itemize}
    \item ein reales Objekt in der physischen Welt,
    \item ein digitales Abbild dieses Objekts in einem virtuellen Raum sowie
    \item eine Schnittstelle, die den Informationsfluss zwischen diesen beiden ermöglicht.
\end{itemize}

Auf Basis des von Grieves entwickelten Informationsmodells hat sich der Begriff des digitalen Zwillings kontinuierlich weiterentwickelt.
Aufgrund verschiedener Fachgebiete und inkonsistenter Definitionen hat sich in der Vergangenheit allerdings eine Vielzahl unterschiedlicher Ausprägungen des Begriffs gebildet.
Diese unterscheiden sich insbesondere in der Tiefe der Datenintegration zwischen dem physischen Objekt und seinem virtuellen Abbild.
Während ein einfacher digitaler Zwilling lediglich ein Modell mit statischen Daten ist, ermöglichen fortgeschrittene Varianten einen bidirektionalen Datenaustausch zwischen physischem und virtuellem Objekt. 

Für ein besseres Verständnis und zur tieferen Klassifizierung ist es zunächst hilfreich, zwischen Typen und Instanzen des digitalen Zwillings zu unterscheiden.
Typen sind allgemeine Abbilder, die grundlegende Eigenschaften und Verhaltensmodelle einer Produktgruppe beschreiben. 
Sie können mit einer Klasse in der Softwareentwicklung verglichen werden, die als Vorlage für konkrete Instanzen dient.
Typen können beispielsweise einen bestimmten Maschinentyp hinsichtlich Aufbau, Struktur oder Schnittstellen beschreiben, ohne dabei Bezug zu einer einzelnen physischen Maschine zu nehmen.

Instanzen hingegen sind einzigartig und beschreiben ein konkretes Produkt, etwa eine Maschine, die eindeutig über eine Seriennummer identifizierbar ist.
Häufig sind sie Ausprägungen eines Typs mit einer Verbindung zu einem realen Objekt, wodurch beispielsweise die Zustandsüberwachung einer Produktionsanlage ermöglicht wird.
Analog zur Softwareentwicklung können sie als instanziiertes Objekt einer Klasse gesehen werden. \cite{ZEISS}

Je nach Art des Informationsflusses sowie dem Grad der Ausprägung der Verbindung zur realen Welt werden Instanzen digitaler Zwillinge häufig in drei Kategorien eingeteilt: das digitale Modell, den digitalen Schatten und den digitalen Zwilling \cite{ClassificationDT}.
Obwohl diese Begriffe im allgemeinen Sprachgebrauch oft synonym verwendet werden, unterscheiden sie sich deutlich hinsichtlich ihrer Funktion und Kopplung zum realen Objekt.

Digitale Modelle sind statische Abbilder physischer Objekte, besitzen jedoch keine direkte Verbindung zur realen Welt. 
Sie werden vor allem zur Veranschaulichung oder Konstruktion genutzt, etwa als 3D-Modell einer Maschine. 
Zwar lassen sich Daten wie Maße oder Materialeigenschaften integrieren, die Eingabe erfolgt jedoch manuell. 
Eine automatische Aktualisierung bei Änderungen am realen Objekt findet nicht statt, sodass das digitale Modell vollständig von seinem physischen Gegenstück getrennt bleibt.

Der digitale Schatten erweitert dieses Konzept um eine unidirektionale Kopplung zum realen Objekt. 
Daten werden hierbei in Echtzeit über geeignete Sensoren erfasst und in das digitale Abbild übertragen, wodurch der aktuelle Zustand des Objekts kontinuierlich nachvollziehbar ist. 
Ein typisches Beispiel ist das Condition Monitoring, bei dem Betriebsdaten einer Maschine digital abgebildet werden. 
Der Informationsfluss erfolgt automatisiert, bleibt jedoch ausschließlich auf eine Richtung beschränkt.

Der digitale Zwilling schließlich ergänzt den digitalen Schatten um eine Rückkopplung zum physischen Objekt. 
Durch diese bidirektionale Verbindung entsteht eine Feedback-Schleife, die es dem virtuellen Abbild ermöglicht, aktiv Einfluss auf das reale System zu nehmen, beispielsweise durch Steuerungsbefehle.
Damit stellt der digitale Zwilling die umfassendste Ausprägung dar und eröffnet Potenziale, die über reine Dokumentation oder Überwachung hinausgehen.
Abbildung~\ref{fig:klassifizierungDT} fasst die drei Kategorien zusammen und verdeutlicht die Unterschiede im Hinblick auf Datenfluss und Kopplung zur realen Welt.

\begin{figure}[htbp]
    \centering
    \includegraphics[width=1\textwidth]{Bilder/klassifizierung_DT.pdf}
    \caption[Klassifizierung des digitalen Zwillings]{Klassifizierung des digitalen Zwillings (in Anlehnung an \cite{ClassificationDT})}
    \label{fig:klassifizierungDT}
\end{figure}
\vspace{-0.5em}

\newpage
Während in der Theorie zwischen digitalem Modell, digitalem Schatten und digitalem Zwilling klar unterschieden wird, werden in der Praxis häufig alle drei Ausprägungen unter dem Begriff digitaler Zwilling zusammengefasst.
Die nachfolgend beschriebenen Einsatzmöglichkeiten und Vorteile beziehen sich daher allgemein auf digitale Zwillinge im industriellen Umfeld, wobei deren konkrete Ausprägung variieren kann.

Bereits in der Entwicklung bieten digitale Zwillinge erhebliche Vorteile, da virtuelle Modelle und Simulationen frühzeitig eingesetzt werden können. 
Dadurch entfällt die Notwendigkeit physischer Prototypen, was Entwicklungszeiten verkürzen und Kosten reduzieren kann. 
In der Produktion ermöglichen sie eine durchgängige Überwachung, Analyse und Optimierung von Fertigungsprozessen auf Basis von Echtzeitdaten. 
Zudem unterstützen sie die virtuelle Inbetriebnahme von Maschinen und können eine Grundlage für vorausschauende Wartung (\ac{pm}) bilden, wodurch sich Stillstandzeiten reduzieren lassen.
Darüber hinaus dienen digitale Zwillinge als zentrale Datenplattform, die Informationen aus unterschiedlichen Quellen zusammenführt und eine konsistente Datenbasis für nachgelagerte Prozesse schafft. \cite{DTForSmartManufacturing}

Die genannten Vorteile verdeutlichen das große Potenzial dieses Konzepts. 
Die praktische Umsetzung erweist sich jedoch oftmals als anspruchsvoll.
Eine zentrale Herausforderung ist die fehlende Interoperabilität zwischen verschiedenen \ac{it}-Systemen.
Sowohl innerhalb eines Unternehmens als auch unternehmensübergreifend bilden sich dadurch häufig voneinander isolierte Datenbestände, die nicht systemübergreifend nutzbar sind.
Solche Informationssilos können die Umsetzung eines konsistenten digitalen Zwillings erheblich erschweren, da die relevanten Informationen und Daten zunächst aus unterschiedlichen Systemen wie \ac{erp}, \ac{mes} oder \ac{cad} zusammengeführt werden müssen.

Hinzu kommt, dass diese Daten häufig in unterschiedlichen, nicht standardisierten Formaten vorliegen, was eine automatisierte Integration zusätzlich erschwert.
Diese Problematik zeigt sich nicht nur innerhalb einzelner Unternehmen, sondern auch entlang der gesamten Wertschöpfungskette, etwa wenn verschiedene Akteure einer Lieferkette heterogene Datenformate und proprietäre Austauschprotokolle verwenden.
Ein digitaler Zwilling, der in einem Unternehmen A erstellt wurde, kann dadurch von einer Anwendung oder einem weiteren digitalen Zwilling eines Unternehmens B nicht unmittelbar interpretiert oder verwendet werden.
Es ist daher essenziell, digitale Zwillinge in einem interope\-rablen Format bereitzustellen, um eine einheitliche Interpretation und Nutzung auch über Unternehmensgrenzen hinweg zu ermöglichen.
\cite{DTandAASConceptsInI4.0}

\newpage
\subsection{Asset Administration Shell}
\label{chap:AAS}

Die \acs{aas} (deutsch: Verwaltungsschale) stellt eine Schlüsselkomponente innerhalb des Referenzarchitekturmodells Industrie 4.0 (\acs{rami}) \cite{RAMI4.0} dar und bildet die Grundlage für die Realisierung interoperabler digitaler Zwillinge. 
Sie wurde maßgeblich von der Plattform Industrie 4.0 entwickelt und im Jahr 2016 erstmals als Bestandteil von \acs{rami} vorgestellt. 
Seit ihrer Einführung wurde sie schrittweise standardisiert und ist mittlerweile in der internationalen Norm IEC 63278-1 \cite{AASIEC63278} verankert.

Die Koordination der Weiterentwicklung erfolgt seit 2020 durch die Industrial Digital Twin Association (\acs{idta}) \cite{IDTA}. 
Ziel der Organisation ist es, den digitalen Zwilling auf Basis der \acs{aas} weiter zu standardisieren und über Open-Source-Softwarelösungen in das industrielle Umfeld zu integrieren. 
Die Struktur ist in mehreren Spezifikationen der \acs{idta} beschrieben.
\acs{aas}-Version 3 stellt den neuesten Entwicklungsstand dar und bildet zugleich die Grundlage für diese Arbeit.

Die zentrale Funktion der \acs{aas} besteht in der digitalen Repräsentation eines Assets. 
Sie verwaltet alle relevanten Daten, Eigenschaften und Funktionen über den gesamten Lebenszyklus hinweg in standardisierter Form. 
Dabei können sowohl physische Objekte (z.~B. Maschinen oder Anlagen) als auch virtuelle Komponenten (z.~B. Software oder Konzepte) abgebildet werden. 
Damit leistet sie einen entscheidenden Beitrag zur Umsetzung digitaler Zwillinge in Industrie 4.0.

Jede \acs{aas} ist eindeutig einem Asset zugeordnet und global identifizierbar. 
Durch die Kombination beider entsteht eine sogenannte Industrie-4.0-Komponente, die in Abbildung~\ref{fig:Industrie4Komponente} dargestellt ist. 
In ihr werden die Informationen eines Assets in verschiedenen Teilmodellen strukturiert, die jeweils spezifische Daten, Dokumente und Funktionen enthalten.
Eine detaillierte Betrachtung des Aufbaus sowie der Themen Informationsaustausch und Sicherheit erfolgt in den nachstehenden Kapiteln.

\vspace{0.25em}
\begin{figure}[htbp]
    \centering
    \includegraphics[width=1\textwidth]{Bilder/I4Komponente/I4KomponenteNeu.pdf}
    \caption[Industrie-4.0-Komponente]{Industrie-4.0-Komponente (Bildbestandteil: \cite{robocellLogo})}
    \label{fig:Industrie4Komponente}
\end{figure}

\subsubsection{Aufbau und Struktur}
Analog zu der in Kapitel \ref{sec: DT} beschriebenen Unterscheidung von Typen und Instanzen digitaler Zwillinge unterscheidet auch die \acs{aas} zwischen diesen beiden Ausprägungen.
Während Typ-\acs{aas} allgemeine Merkmale und Strukturen eines Produkttyps beschreiben, sind Instanz-\acs{aas} konkreten physischen oder virtuellen Objekten zugeordnet und enthalten spezifische Informationen wie Seriennummer, Zustand oder Standort.

Die Struktur der \acs{aas} wird durch ein Metamodell \cite{SpezifikationPart1} vorgegeben, das in \ac{uml}-Klassendiagrammen spezifiziert ist.
Diese Diagramme definieren die zulässigen Elemente des Informationsmodells sowie deren Beziehungen untereinander und bilden damit die Grundlage für eine standardisierte Repräsentation eines Assets.

Bestimmte Aspekte werden gemäß dieser Spezifikation in Submodellen organisiert.
Diese lassen sich als Container verstehen, die jeweils einen bestimmten Bereich des Assets abbilden, beispielsweise das Typenschild, technische Stammdaten, Wartungsinformationen und Zustandswerte einer Maschine.  
Die Auswahl und Ausgestaltung der Submodelle sind domänenspezifisch und richten sich stark nach dem jeweiligen Anwendungsfall.
Jede \acs{aas} kann beliebig viele Submodelle umfassen, die bei Bedarf auch erweitert werden können.

Die Daten innerhalb eines Submodells werden in verschiedenen Submodellelementen strukturiert.
Diese umfassen Dateneigenschaften, Operationen sowie weitere Strukturelemente, die für eine umfassende Beschreibung eines digitalen Modells erforderlich sind.
Das RelationshipElement erlaubt beispielsweise die Modellierung von Beziehungen, während das ReferenceElement zur Referenzierung interner oder externer Inhalte dient.

Das vermutlich am häufigsten verwendete Datenelement ist das Submodellelement 
\linebreak Property.
Es lässt sich mit einer Variablen aus der Softwareentwicklung vergleichen, da es einfache Merkmale wie etwa einen Namen oder eine Seriennummer abbildet und über einen definierten Datentyp wie String, Integer oder Boolean verfügt.

Zur besseren Veranschaulichung dient Abbildung \ref{fig:MetamodellAAS}, die eine vereinfachte Darstellung des zugrunde liegenden Metamodells zeigt.

Wichtig ist, dass sowohl die \acs{aas} selbst als auch ihre Submodelle global eindeutig identifizierbar sind.
Dies wird durch die Verwendung eindeutiger Identifikatoren (\acsp{id}) der Klasse Identifiable wie einer URI (Uniform Resource Identifier) oder \acs{irdi} (International Registration Data Identifier) sichergestellt.
Für die Elemente innerhalb eines Submodells ist hingegen eine lokale Kennung ausreichend. 
Dies erfolgt anhand einer idShort der Klasse Referable, die einen kurzen, aussagekräftigen Namen enthält.

\newpage
\begin{figure}[htbp]
    \centering
    \includegraphics[width=1\textwidth]{Bilder/Metamodell/MetamodellFarbig.pdf}
    \caption[Vereinfachtes Metamodell der \acs{aas}]{Vereinfachtes Metamodell der \acs{aas} (in Anlehnung an \cite{SpezifikationPart1})}
    \label{fig:MetamodellAAS}
\end{figure}

Die eindeutige semantische Beschreibung aller Submodelle und ihrer Elemente ist essenziell, um ein einheitliches Verständnis zwischen verschiedenen Systemen sicherzustellen.
Zu diesem Zweck dient die sogenannte semanticId der Klasse HasSemantics, die eine semantische Referenz enthält.
Diese verweist entweder auf einen externen Standard oder auf eine lokale \ac{cd}, die direkt in der \acs{aas} eingebettet ist.
Ein häufig verwendeter externer Standard ist zum Beispiel ECLASS, der auf der Norm IEC 61360 \cite{ECLASSIEC61360} basiert.

Zur formalen Beschreibung solcher Concept Descriptions stellt die \acs{idta} standardisierte Data Specification Templates gemäß IEC 61360 bereit \cite{SpezifikationPart3a}.
Diese liefern ein strukturiertes Modell zur Beschreibung technischer Merkmale.
Darin enthalten sind unter anderem Definitionen, Einheiten, Wertebereiche, zulässige Werte sowie externe Referenzen, die bestimmten Submodellen oder Submodellelementen zugeordnet werden können.

Um die Erstellung von Submodellen zu erleichtern und gleichzeitig Interoperabilität zu gewährleisten, stellt die \acs{idta} standardisierte Submodellvorlagen -- sogenannte \acp{smt} -- zur Verfügung.
Diese decken eine Vielzahl industrieller Anwendungsfälle ab und werden kontinuierlich weiterentwickelt und ergänzt.
Bereits verfügbare Templates umfassen unter anderem Submodelle wie beispielsweise das digitale \linebreak
Typenschild und den \ac{cf}.
Alle Submodellelemente innerhalb dieser Vorlagen sind in Verbindung mit dem ECLASS-Standard einheitlich semantisch beschrieben.
Die Templates können über ein zentrales Repository \cite{idtaTemplates} bezogen werden und bilden die Basis für eine interoperable Datenstruktur.

\subsubsection{Informationsaustausch}
Der Austausch von Informationen über die \acs{aas} kann auf unterschiedliche Weise erfolgen.
Die einfachste Möglichkeit besteht im Dateiaustausch. Hierfür wurden speziell für die \acs{aas} sogenannte AASX-Dateien \cite{SpezifikationPart5} entwickelt, die den einfachen Austausch statischer \acs{aas} (Typ-1-\acs{aas}) ermöglichen.
Dabei werden sämtliche Daten, Beziehungen, Strukturen sowie zugehörige Dateien der \acs{aas} serialisiert und in ein AASX-ZIP-Dateiformat gespeichert. 
Diese Datei kann anschließend über ein digitales Medium, etwa per E-Mail oder über eine Cloud-Plattform, weitergegeben werden. 

Eine Typ-2-\acs{aas} hingegen wird von einer Laufzeitumgebung gehostet, wodurch ein direkter und dynamischer Zugriff auf ihre Inhalte ermöglicht wird. 
Die Spezifikation Part 2: Application Programming Interfaces \cite{SpezifikationPart2} beschreibt hierfür nicht nur standardisierte Schnittstellen, sondern auch ein ganzheitliches System für das Verwalten, Bereitstellen und Auffinden der \acs{aas}.
Repositories dienen dabei als zentraler Speicherort für die Inhalte einer \acs{aas}, einschließlich ihrer Submodelle und Concept Descriptions.
Die Aufgabe der Verwaltung und Registrierung übernehmen sogenannte Registries.
Sie ermöglichen das systemweite Auffinden von \acs{aas} und stellen sicher, dass diese eindeutig referenzierbar sind.

Ergänzend dazu bieten Discovery Services eine erweiterte Suchfunktionalität, indem sie Beziehungen verschiedener Entitäten mittels verschiedener Schlüsselwertpaare speichern.
Eine \acs{aas} kann so zum Beispiel logisch mit einer Asset-\acs{id} verknüpft und somit schnell innerhalb komplexer Systeme identifiziert werden.
Der Zugriff auf diese Systeme bzw. ihre Inhalte wird in Form von Schnittstellen standardisiert, wodurch eine hohe Interoperabilität gewährleistet wird.
Ein besonderer Fokus liegt dabei auf der Nutzung von \ac{http} gemäß dem \acs{rest}-Architekturstil (Representational State Transfer), der eine strukturierte Kommunikation über Methoden wie GET, POST, PUT oder DELETE ermöglicht.

Die fortschrittlichste Form des Informationsaustausches stellt die Peer-to-Peer-Kommuni\-kation dar, bei der Industrie-4.0-Komponenten eigenständig über die Industrie-4.0-Sprache \cite{I4Sprache} miteinander kommunizieren (Typ-3-\acs{aas}).

\subsubsection{Sicherheit}
\label{sec: Sicherheit}
Gerade wenn Informationen aus der \acs{aas} über die Grenzen des eigenen Unternehmens hinweg bereitgestellt werden, ist es besonders wichtig, dass die enthaltenen Daten geschützt sind. 
Die neueste Spezifikation Part 4: Security \cite{SpezifikationPart4} der \acs{idta} liefert hierfür die technische und konzeptionelle Grundlage.
Sie beschreibt, wie Zugriffe auf Daten in der \acs{aas} sicher gesteuert werden können, insbesondere in vernetzten Umgebungen wie Datenräumen.

Zum Einsatz kommen Dienste wie ein Identity Provider zur Authentifizierung oder ein Policy Service zur Durchsetzung von Richtlinien.
Die Sicherheit wird mithilfe eines attributbasierten Zugriffsmodells (\ac{abac}) gewährleistet.
Bei jeder Anfrage auf bestimmte Objekte innerhalb der \acs{aas} wird anhand verschiedener Merkmale (Attribute) geprüft, ob ein Zugriff erlaubt ist.
Dazu zählen sogenannte Subjekt\-attribute (also wer die Anfrage stellt), Objektattribute (z.~B. welches Submodell oder welches Submodellelement betroffen ist), die gewünschte Aktion (z.~B. Lesen oder Schreiben) sowie kontextbezogene Bedingungen (z.~B. Zeitpunkt der Anfrage oder Zustand des Systems).

Die zur Prüfung notwendigen Informationen liefert ein Token, das vom Identity Provider bereitgestellt wird. 
Die Spezifikation sieht die Nutzung sogenannter \acp{jwt} vor.
Die darin enthaltenen Attribute werden vom Policy Service mit den dort hinterlegten Zugriffsrichtlinien abgeglichen und basierend darauf eine Zugriffsentscheidung getroffen.
Ein besonderer Vorteil des \acs{abac}-Modells liegt dabei in seiner hohen Flexibilität. 
Rollen können ebenfalls als Attribute behandelt werden, wodurch sich auch problemlos rollenbasierte Zugriffskonzepte (\ac{rbac}) umsetzen lassen. 

Die beschriebenen Kontrollmechanismen lassen sich nicht nur auf die Inhalte der \acs{aas} selbst, sondern insbesondere auch auf die Schnittstellen von Registries und Repositories anwenden.
So kann beispielsweise sichergestellt werden, dass nur autorisierte Systeme Zugriff auf ein bestimmtes Submodell erhalten oder nur bestimmte Nutzergruppen neue \acs{aas}-Instanzen registrieren können.
Diese Sicherheitskonzepte sind jedoch noch vergleichsweise neu und müssen in der Praxis erst noch weiter erprobt werden.
Erste Referenzimplementierungen liegen zwar bereits in Form rollenbasierter Zugriffskontrollen vor, eine vollständige Integration des \acs{abac}-Ansatzes steht jedoch noch aus.


\subsection{Digitaler Produktpass}
Der \acs{dpp} ist ein zentrales Instrument der \acs{eu} zur Förderung einer nachhaltigen und digitalen Transformation. 
Ziel ist es, die Transparenz über ökologische Eigenschaften von Produkten wie verwendete Materialien, Recycelbarkeit oder die CO\textsubscript{2}-Bilanz deutlich zu verbessern.
Hierzu müssen produktspezifische Daten über den gesamten Lebenszyklus hinweg erfasst und in einem menschen- und maschinenlesbaren Format bereitgestellt werden. \cite{DPPEinführung}
Langfristig soll dies den Übergang zu einer Kreislaufwirtschaft innerhalb der \acs{eu} unterstützen.

Das Konzept des \acs{dpp} wurde erstmals 2019 im Rahmen des European Green Deal von der Europäischen Kommission vorgestellt \cite{GreenDeal}.
Im Zuge der Ökodesign-Verordnung (\ac{espr}) \cite{ESPR} wird er aktuell als verpflichtendes Mittel für zahlreiche Produktgruppen eingeführt.
Als erste konkrete Anwendung wird der \acs{dpp} im Jahr 2027 erstmals für Batterien verpflichtend.
Weitere Produktkategorien, darunter auch die Elektroindustrie sowie der Maschinen- und Anlagenbau, sollen in den nächsten Jahren folgen.

Die Bereitstellung der Produktpässe erfolgt gemäß den Anforderungen der \acs{espr} in elektronischer Form. 
Dabei müssen diese interoperabel sein.
Je nach Art der Information werden verschiedene Zugriffsrechte für unterschiedliche Interessengruppen eingeführt. 
Damit soll der Schutz von geistigem Eigentum sichergestellt werden.
Die Daten sollen auf einem zentralen Server gespeichert und über eine zentrale Registry verwaltet werden, in der die verschiedenen \acsp{dpp} registriert sind.
\cite{CIRPASS}

Während die regulatorischen Rahmenbedingungen mehr oder weniger final ausgearbeitet sind, bleibt die Frage der konkreten technologischen Umsetzung.
Eine dezentrale Lösung bildet der von der \acs{zvei} vorgestellte Digitale Produktpass für Industrie 4.0 (\acs{dpp40}) \cite{DPP40}.
Dieser basiert auf zwei etablierten Standards: zum einen dem digitalen Typenschild, zum anderen der \acs{aas} (siehe auch Kapitel~\ref{chap:AAS}).
Das digitale Typenschild ermöglicht, gemäß der Norm IEC 61406 \cite{TypenschildIEC61406-1}, die eindeutige Identifikation von Produkten über ihre einzigartige Asset-\acs{id}.
Typischerweise wird diese in Form eines maschinenlesbaren Links oder eines Quick Response Codes (\acs{qr}-Code) an einem Produkt angebracht, der direkt zum zugehörigen \acs{dpp} führt.

\clearpage
Organisiert werden die Daten im \acs{dpp40} in verschiedenen Submodellen der \acs{aas}. 
Standardisierte \acsp{smt} wie das digitale Typenschild, Übergabedokumentation oder der \acs{cf} helfen bei der Umsetzung der im \acs{dpp} geforderten Daten.
Darüber hinaus können auch zusätzliche, nicht verpflichtende Informationen integriert werden, sofern sie für bestimmte 
\linebreak
Stakeholder einen Mehrwert bieten.
Der Zugriff auf die Daten ist über ein webbasiertes Portal vorgesehen. 

Verschiedene Interessengruppen erhalten dabei unterschiedliche Zugriffsrechte. 
Hierfür werden bestimmte Submodelle gezielt für unterschiedliche Gruppen freigegeben oder deren Zugriff eingeschränkt. 
Während beispielsweise das Typenschild öffentlich zugänglich ist, werden sensible Informationen wie technische Dokumentationen und sicherheitsrelevante Details ausschließlich bestimmten autorisierten Gruppen zugänglich gemacht.

Das Konzept der \acs{zvei} sieht darüber hinaus vor, dass Unternehmen ihre \acsp{dpp} entgegen den Anforderungen der \acs{espr} dezentral in einem eigenen Repository verwalten. 
Dieses kann entweder vom produzierenden Unternehmen selbst oder von Dritten, etwa Cloud-Dienstleistern, im Auftrag betrieben werden. 
Ziel ist es, Unternehmen die Möglichkeit zu geben, ihre Daten bei Bedarf eigenständig zu aktualisieren und gleichzeitig die Kontrolle über sensible Informationen zu behalten.

Zur Koordination dieser dezentralen Systeme ist eine zentrale Registry vorgesehen, in der alle Repositories registriert werden.
Über diese können interessierte Akteure relevante Server identifizieren und gezielt auf freigegebene Submodelle eines Produktpasses zugreifen.
So wird sichergestellt, dass trotz der dezentralen Struktur eine durchgängige Interoperabilität gewährleistet ist, wie sie für die Umsetzung des \acs{dpp} auf europäischer Ebene erforderlich ist.

\subsection{robocell}
Die robocell ist eine von groninger in Zusammenarbeit mit SKAN entwickelte Maschinenlinie zur aseptischen Abfüllung von genesteten Spritzen, Zylinderampullen und Vials.
Sie zeichnet sich dadurch aus, dass alle Prozesschritte vollständig automatisiert ablaufen.
Durch den gezielten Einsatz von Robotern kann der menschliche Eingriff auf ein Minimum reduziert werden, wodurch maximale Sicherheit, Flexibilität und Effizienz im Abfüllprozess gewährleistet werden \cite{RobocellWebsite}.

Die Linie besteht aus mehreren modular aufgebauten Einzelmaschinen, die jeweils spezifische Aufgaben entlang des Produktionsprozesses übernehmen. 
Im Rahmen dieser Arbeit liegt der Fokus auf dem in Abbildung~\ref{fig:robocell} gezeigten hochautomatisierten Abfüll- und%
\pagebreak
~Verschließmodul, das für das vollautomatisierte Abfüllen und Verschließen von Behältnissen verantwortlich ist.

\begin{figure}[htbp]
    \centering
    \includegraphics[width=0.5\textwidth]{Bilder/robocell/filling_closing_module.png}
    \caption[robocell: Abfüll- und Verschließmodul]{robocell: Abfüll- und Verschließmodul (Quelle: \cite{RobocellBetriebsanleitung})}
    \label{fig:robocell}
\end{figure}

\vspace{-0.25em}
\subsection{Technologische Grundlagen}
Im Folgenden werden die technologischen Grundlagen erläutert, die für das Verständnis und die Umsetzung dieser Arbeit besonders relevant sind.
\subsubsection{AASX Package Explorer}
Der AASX Package Explorer, nachfolgend als Package Explorer bezeichnet, wurde als Referenzimplementierung für die \acs{aas} gemäß den Spezifikationen der \acs{idta} entwickelt.
Das Tool ist als Open-Source-Software \cite{AASXPackageExplorer} verfügbar und ermöglicht das Erstellen und Bearbeiten von \acs{aas} im standardisierten AASX-Dateiformat.

Der Package Explorer verfügt dabei über eine benutzerfreundliche grafische Oberfläche (siehe Abbildung \ref{fig:AASXPackageExplorer}), die insbesondere Einsteigern den Zugang zur Modellierung erleichtert.
Dabei können Submodelle, Eigenschaften, semantische Referenzen sowie Metadaten strukturiert definiert und verwaltet werden.
Gleichzeitig bietet der Package Explorer auch erweiterte Funktionen, wie das Erstellen von \acsp{smt}, wodurch er sich auch für den professionellen Einsatz eignet.

Neben der lokalen Modellierung erlaubt das Tool ebenfalls die Verbindung zu einem AAS-Server über standardisierte Schnittstellen (z.B. \acs{opcua} oder \acs{http}/\acs{rest}).
Dies ermöglicht den Betrieb von \acs{aas} in verteilten Systemen.
Besonders geeignet hierfür%
\pagebreak
~ist der Referenzserver des Eclipse-AAS-Projekts \cite{AASXServer}, der das Hosten und Bereitstellen von AASX-Paketen ermöglicht sowie eine nahtlose Integration mit dem Package Explorer erlaubt.

\begin{figure}[htbp]
    \centering
    \includegraphics[scale=0.765]{Bilder/ModellierungAAS/Final/Grundlagen_PE.PNG}
    \caption[Benutzeroberfläche des Package Explorers]{Benutzeroberfläche des Package Explorers} 
    \label{fig:AASXPackageExplorer}
\end{figure}

\subsubsection{Eclipse BaSyx }
Eclipse BaSyx ist eine vom Fraunhofer-Institut für Experimentelles Software Engineering entwickelte Open-Source-Plattform für die Realisierung von Industrie-4.0-Anwendungen.
Mittlerweile wird das Projekt unter dem Dach der Eclipse Foundation weitergeführt.
Der Fokus liegt auf einer einfachen Umsetzung einer Infrastruktur zur Erstellung und Verwaltung digitaler Zwillinge auf Basis der \acs{aas}.
Die Software steht dabei allen Interessenten frei zur Verfügung.

Die Softwarearchitektur basiert auf einer Vielzahl von Standardkomponenten (Off-the-Shelf), die alle als Docker-Container frei zugänglich sind und somit eine nahtlose Integration in bestehende Docker-Umgebungen erlauben.
Eine der wichtigsten Komponenten ist die Registry. 
Sie ist, genau wie alle anderen Komponenten, auf den Spezifikationen der \acs{aas} aufgebaut, insbesondere auf der Spezifikation Part 2: Application Programming Interfaces \cite{SpezifikationPart2}.
In ihr können neue \acs{aas} registriert und bereits vorhandene \acs{aas} anhand ihrer eindeutigen Kennung gesucht werden.
Sie bildet damit die zentrale Anlaufstelle für Geräte und Anwendungen innerhalb des BaSyx-Systems.
Analog dazu existiert eine separate Registry für die Verwaltung von Submodellen.

\newpage
Die eigentlichen Daten werden in der sogenannten \acs{aas} Environment gespeichert und organisiert.
Sie umfasst Repositories für \acs{aas}, Submodelle und Concept Descriptions.
In der Regel ist eine Datenbank, standardmäßig eine MongoDB als persistenter Speicher hinterlegt.
Wie auch alle anderen Komponenten stellen diese Repositories standardisierte Schnittstellen basierend auf der \ac{api}-Spezifikation zur Verfügung.
Dies erlaubt z.~B. das Abfragen, Erstellen oder Aktualisieren einer \acs{aas} samt ihrer Submodelle und Inhalte.
Alle verfügbaren Endpunkte dieser Schnittstellen können unter anderem in der automatisch generierten Swagger-Dokumentation eingesehen und ausgeführt werden. 
Eine Auswahl der wichtigsten Endpunkte der AAS Environment ist in Tabelle \ref{tab:aas_endpoints} dargestellt.

% \vspace{1em}
{\small
\begin{longtblr}[
  label = tab:aas_endpoints,
  caption = {REST-Endpunkte in Eclipse BaSyx},
  entry = REST-Endpunkte in Eclipse BaSyx
]{
  colspec = {c l X},
  rowhead = 1,
  vlines,
  hlines
}
\textbf{Methode} & \textbf{Endpunkt} & \textbf{Beschreibung} \\
\textbf{\textit{\textcolor{swaggerget}{GET}}} & \texttt{/shells} & Liste aller AAS abrufen \\
\textbf{\textit{\textcolor{swaggerget}{GET}}} & \texttt{/shells/\{aasIdentifier\}} & Bestimmte AAS anzeigen \\
\textbf{\textit{\textcolor{swaggerget}{GET}}} & \texttt{/submodels} & Liste aller Submodelle aufrufen \\
\textbf{\textit{\textcolor{swaggerpost}{POST}}} & \texttt{/shells} & Neue AAS erstellen \\
\textbf{\textit{\textcolor{swaggerpost}{POST}}} & \texttt{/submodels} & Neues Submodell erstellen \\
\textbf{\textit{\textcolor{swaggerdelete}{DELETE}}} & \texttt{/shells/\{aasIdentifier\}} & AAS löschen \\
\end{longtblr}
}
\vspace{-0.75em}

Im BaSyx-System ermöglicht ein Discovery Service zudem die Verknüpfung physischer Assets mit ihren zugehörigen \acs{aas}.
Dies ist insbesondere für die Abbildung von hierarchischen Strukturen wie Stücklisten (\ac{bom}) von großer Bedeutung.
Ein übergeordnetes Asset (z.B. Maschine) kann so mit untergeordneten Komponenten (z.B. Antrieb, Sensoren) logisch über deren AAS verbunden werden.
Einträge in den Discovery Service müssen derzeit allerdings noch manuell über die \acs{api} vorgenommen werden.

Zur benutzerfreundlichen Visualisierung und Interaktion kann die sogenannte AAS Web UI genutzt werden.
Die webbasierte Benutzeroberfläche, wie in Abbildung \ref{fig:BasyxWebUI} dargestellt, wurde mit dem JavaScript-Framework Vue.js entwickelt und kommuniziert über die standardisierte \acs{rest}-API mit den zentralen Komponenten der BaSyx-Plattform, darunter die Repositories, Registries und der Discovery Service.
Sie zeigt alle registrierten \acs{aas} in einer Liste an und bietet die Möglichkeit, einzelne \acs{aas} in einer Baumstruktur sowohl zu visualisieren als auch zu bearbeiten. 

Ein weiteres zentrales Merkmal der AAS Web UI ist der sogenannte AAS Viewer.
In nachfolgender Abbildung ist dieser auf der rechten Seite angeordnet.
Er erlaubt die Visualisierung von Submodellen und deren Elementen anhand ihrer semanticId. 
Hierfür stehen verschiedene vordefinierte Plugins zur Verfügung, die bestimmte Submodelle, wie beispielsweise das Typenschild oder hierarchische Strukturen, grafisch darstellen.
Da die Lösung Open Source ist, besteht zudem die Möglichkeit, eigene benutzerdefinierte Plugins für weitere Submodelle zu erstellen. \cite{BaSyxWiki,BaSyxEclipse} 

\vspace{0.5em}
\begin{figure}[htbp]
    \centering
    \includegraphics[width=1\textwidth]{Bilder/AASWebUZIGrundlagen.png}
    \caption[Benutzeroberfläche der AAS Web UI]{Benutzeroberfläche der AAS Web UI}
    \label{fig:BasyxWebUI}
\end{figure}


\subsubsection{OPC Unified Architecture}
\acs{opcua} ist ein plattformübergreifender Kommunikationsstandard, der speziell für die Anforderungen der industriellen Automatisierung entwickelt wurde.
Ziel ist ein herstellerübergreifender, sicherer und standardisierter Datenaustausch.
In \acs{rami} \cite{RAMI4.0} wird \acs{opcua} als empfohlener Standard für die Kommunikationsschicht definiert und bildet damit die Grundlage für die Interoperabilität zwischen Maschinen, Anlagen und \acs{it}-Systemen verschiedener Hersteller.

Die grundlegende Idee von \acs{opcua} besteht darin, dass ein Maschinenhersteller einen \acs{opcua} Server bereitstellt, der einen standardisierten und herstellerunabhängigen Zugriff auf eine Maschine ermöglicht.
Der Server dient hierbei als zentrale Schnittstelle zur Außenwelt. Er implementiert den OPC Standard und stellt strukturierte Informationen sowie Zugriffsmöglichkeiten auf Maschinenzustände und -daten bereit.
Im Inneren kommuniziert der Server dabei über ein herstellerspezifisches, proprietäres Protokoll mit der Steuerung.
Zum Auslesen oder Austauschen dieser Daten wird ein \acs{opcua} Client benötigt. Dieser agiert als Kommunikationspartner des Servers, stellt die Verbindung her und ermöglicht den bidirektionalen Datentransfer. \cite{OPCUA}
\section{Entwicklung}
\subsection{Konzeptionierung des digitalen Zwilling}
\subsubsection{Identifikation relevanter Datenquellen}
\subsubsection{Auswahl geeigneter Teilmodelle}
\subsection{Modellierung mit der Verwaltungsschale}
\subsubsection{Umsetzung mit dem AASX Package Explorer}
\subsubsection{Validieren und Testen}
\subsection{Technische Integration}
\subsubsection{Bereitstellung der Verwaltungsschalen}
\subsubsection{Datenzugriff über standardisierte Schnittstellen}
\subsubsection{Integration und Verarbeitung von Echtzeitdaten}
\paragraph{Anbindung OPC UA mit Databridge}
\paragraph{Zeitreihendaten mit Datenbank}
\subsection{Anwendungsfall Digitaler Produktpass}
\subsubsection{Beschreibung des Use Case}
\subsubsection{Umsetzung mit dem Teilmodell Carbon Footprint}
\subsection{Anwendungsfall Automatisierte Generierung von Verwaltungsschalen}
\subsubsection{Erstellen von Templates}
\subsubsection{Ausfüllen mit strukturierten Daten}
\subsubsection{Automatisches Bereitstellen der AAS mit der Rest API}
\subsubsection{Einsatz von KI}
\section{Ergebnisse}
\subsection{AAS-Demonstrator für die robocell}
\subsubsection{Systemarchitektur}
\subsubsection{Eingesetzte Teilmodelle}
\subsubsection{Herausforderungen bei der Erstellung}
\subsection{Anwendungsfall Digitaler Produktpass}
\subsubsection{Implementierungskonzept}
\subsubsection{Dynamische Berechnung des PCF}
\subsection{Anwendungsfall automatisierte Generierung der AAS}
\subsection{Einsatzmöglichkeiten von KI im Kontext der Verwaltungsschale}
\subsubsection{Generierung von Verwaltungsschalen}
\subsubsection{Anomaliererkennung}
\subsubsection{Weiterführende Einsatzmöglichkeiten}
\subsection{Evaluierung eingesetzter Tools und Software}
\subsubsection{AASX Package Exlporer}
\subsubsection{Eclipse AASX Server}
\subsubsection{BaSyx}
\subsubsection{Mnestix Browser}
\newpage
\section{Zusammenfassung und Ausblick}
\label{sec:Zusammenfassung}
\subsection{Zusammenfassung der Arbeit}
\subsection{Handlungsempfehlung für groninger}
+ kurzfristig nicht optimal
+ PLM-System ist nicht dafür ausgelget mit der AAS zu arbeiten - in Zukunft wird aber auf Contact Elements umgestiegen dann könnte es durchaus sinnn machen
+ Daten interoperabel Bereitstellen allles an einem ort gebündelt zum Beispiel wenn Betreiber eigenen Digitalen Zwilling, dann aber mit 3-D Simulationen
+ Iwan wird der digitale Produktpass kommen, dafür wäre es gut geeignet
\subsection{Ausblick auf zukünftige Entwicklungen}



%%%%---------------------------------
% \section{Zusammenfassung und Ausblick}
% \label{sec:Zusammenfassung}

% \subsection{Zusammenfassung der Arbeit}
% Ziel dieser Arbeit war es, am Beispiel des Abfüll- und Verschließmoduls der robocell-Linie das Potenzial der Asset Administration Shell (\acs{aas}) für die Modellierung und Nutzung eines digitalen Zwillings zu untersuchen.  
% Dazu wurden zunächst theoretische Grundlagen betrachtet und die eingesetzten Werkzeuge vorgestellt. Im praktischen Teil erfolgte die prototypische Implementierung einer Haupt-\acs{aas}, die sowohl statische als auch dynamische Submodelle umfasst und über OPC~UA mit realen Prozessdaten verknüpft ist. Ergänzend wurde ein \acs{ki}-basiertes Verfahren zur Anomalieerkennung entwickelt.  
% Darüber hinaus wurden zwei praxisnahe Anwendungsfälle exemplarisch umgesetzt: zum einen die Abbildung eines digitalen Produktpasses einschließlich PCF-Berechnung und differenzierter Zugriffsrechte, zum anderen die automatisierte Generierung von \acs{aas}.  
% Die Ergebnisse zeigen, dass die \acs{aas} ein vielversprechendes Konzept zur interoperablen und standardisierten Beschreibung von Maschinen und Komponenten darstellt. Insbesondere die Erweiterbarkeit durch Submodelle ermöglicht eine flexible Anpassung an unterschiedliche Anforderungen. Zugleich wurde deutlich, dass aktuelle Tools wie Eclipse BaSyx und der Package Explorer zwar eine wertvolle Basis bilden, in ihrer Praxistauglichkeit jedoch noch Einschränkungen aufweisen.

% \subsection{Handlungsempfehlung für groninger}
% Für groninger bietet der Einsatz der \acs{aas} eine vielversprechende Perspektive, um Transparenz, Nachverfolgbarkeit und Interoperabilität im Maschinen- und Anlagenbau zu stärken.  
% Kurzfristig empfiehlt es sich, den Ansatz in Pilotprojekten weiter zu erproben, um Erfahrungen mit Modellierung, Integration und Toolunterstützung zu sammeln. Mittelfristig sollte geprüft werden, wie die \acs{aas} in bestehende Systeme wie das PLM integriert und mit internen Datenmodellen verknüpft werden kann.  
% Besonderes Augenmerk sollte dabei auf regulatorische Entwicklungen gelegt werden. Der digitale Produktpass wird in den kommenden Jahren verpflichtend, sodass eine frühzeitige Auseinandersetzung mit entsprechenden Submodellen (z.\,B. Carbon Footprint) strategische Vorteile bietet. Auch die Nutzung von KI-Verfahren zur Zustandsüberwachung und Optimierung eröffnet Potenziale für Effizienzsteigerungen, die in Zukunft stärker verfolgt werden sollten.

% \subsection{Ausblick auf zukünftige Entwicklungen}
% Die weitere Entwicklung der \acs{aas} wird maßgeblich durch die Arbeit von Standardisierungsgremien wie der \acs{idta} geprägt. Neue Submodelle, eine stärkere Toolunterstützung sowie die Ausweitung auf gesamte Lieferketten werden die Einsatzmöglichkeiten in den kommenden Jahren erweitern.  
% Langfristig ist davon auszugehen, dass die \acs{aas} eine zentrale Rolle in digitalen Ökosystemen einnimmt und damit nicht nur einzelne Maschinen, sondern komplette Produktionsanlagen und Supply Chains digital abbildet. In diesem Kontext gewinnen auch Ansätze wie rollen- oder attributbasierte Zugriffskontrolle, erweiterte KI-Methoden zur Prozessanalyse sowie die Einbindung von Nachhaltigkeitskennzahlen zunehmend an Bedeutung.  
% Damit eröffnen sich für Unternehmen, die frühzeitig Erfahrungen mit der \acs{aas} sammeln, erhebliche Chancen, um künftige regulatorische Anforderungen zu erfüllen, interne Prozesse zu optimieren und sich technologisch im Wettbewerb zu positionieren.

Zur Verwendung des Glossars \gls{SPS} verwenden.
%=======================ENDE HAUPTTEIL====================================%

\clearpage
\newpage
%\pagenumbering{Roman}
%\setcounter{page}{5}
%\pagestyle{plain} %nur Seitenzahl in der Fußzeile (LaTeX-Standard)

\phantomsection \addcontentsline{toc}{section}{Glossar und Abkürzungsverzeichnis}
\renewcommand \refname{Glossar} \section*{Glossar und Abkürzungsverzeichnis}
\renewcommand{\glossarysection}[2][]{}
\printglossary[type=main,nonumberlist]


\newpage
\printbibliography
% \bibliographystyle{abbrvnat}
% \bibliography{references}

\begin{appendix}

\section{Anhang 1}
Hier Anhang einfügen

\end{appendix}

\end{document}
