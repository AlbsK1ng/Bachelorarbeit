\newpage
\section{Zusammenfassung und Ausblick}
\label{sec:Zusammenfassung}

Auf Grundlage der in den vorhergehenden Kapiteln dargestellten Ergebnisse werden im Folgenden die zentralen Erkenntnisse der Arbeit zusammengefasst. 
Daraufhin wird eine Handlungsempfehlung für das Unternehmen groninger abgeleitet, die den möglichen Einsatz und Nutzen der AAS bewertet. 
Den Abschluss bildet ein Ausblick, der mögliche Weiterentwicklungen und künftige Ergänzungen dieser Arbeit aufzeigt.

\subsection{Zusammenfassung der Arbeit}

Ausgehend von den theoretischen Grundlagen wurde ein Demonstrator entwickelt, mit dem sich die Umsetzung digitaler Zwillinge auf Basis der \acs{aas} erproben ließ.
Anhand statischer Submodelle wurde gezeigt, wie Stammdaten einer Maschine interoperabel und nachvollziehbar in der \acs{aas} abgebildet werden können.
Darauf aufbauend folgte eine dynamische Erweiterung, bei der die \acs{aas} als Typ-2-Ausprägung in einer Laufzeitumgebung bereitgestellt wurde und die Integration von Prozess- und Zeitreihendaten ermöglichte.
Dadurch entstand ein vollwertiger digitaler Zwilling, der nicht nur statische Informationen bereitstellt, sondern auch den aktuellen Betriebszustand der Maschine widerspiegelt.

Ferner wurde ein \acs{ki}-Modell zur Anomalieerkennung konzipiert und prototypisch umgesetzt, das auf einem Autoencoder basiert und Temperaturdaten nutzt, die im digitalen Zwilling enthalten sind.
In Tests mit synthetisch erzeugten Anomalien gelang eine zuverlässige Erkennung, womit das Potenzial zur Optimierung der Anlagenverfügbarkeit und zur Reduzierung ungeplanter Stillstandzeiten aufgezeigt wurde.

Darüber hinaus wurden zwei praxisnahe Anwendungsfälle realisiert.
Der Anwendungsfall des \acs{dpp} zeigte, dass Nachhaltigkeitsinformationen in Form von \acs{pcf}-Werten transparent abgebildet und über Eclipse BaSyx rollenbasiert bereitgestellt werden können.
Dies bestätigte, dass die \acs{aas} eine vielversprechende Grundlage für die praktische Umsetzung des \acs{dpp} gemäß dem von \acs{zvei} und \acs{idta} entwickelten Konzept des \acs{dpp40} ist.
Ergänzend ließ sich mit dem Anwendungsfall der automatisierten Generierung von \acs{aas} demonstrieren, wie sich der manuelle Modellierungsaufwand reduzieren lässt und dadurch die effiziente Erstellung von \acs{aas}-Instanzen in der industriellen Praxis möglich wäre.

\newpage
\subsection{Handlungsempfehlung für groninger}

Aus den in dieser Arbeit gewonnenen Erkenntnissen ergibt sich für groninger, dass die AAS ein geeignetes Instrument darstellt, um Maschinendaten strukturiert, interoperabel und maschinenlesbar bereitzustellen. 
Dadurch entsteht eine konsistente Informationsbasis, die sowohl intern genutzt als auch standardisiert an externe Partner weitergegeben werden kann. 
So ließe sich beispielsweise die AAS einer Maschine problemlos an einen Betreiber übermitteln, wenn dieser einen eigenen digitalen Zwilling aufbauen möchte.

In der praktischen Anwendung ist der Einsatz der \acs{aas} allerdings noch eingeschränkt, da das bestehende PLM-System Agile keine direkte Integration unterstützt. 
Die Erstellung einzelner AAS müsste derzeit manuell erfolgen und wäre mit hohem Aufwand verbunden.
Mit dem geplanten Umstieg auf die CIM Database von CONTACT Software eröffnen sich für groninger jedoch mittelfristig aussichtsreiche Perspektiven, diesen Prozess deutlich zu vereinfachen. 
Über CONTACT Elements for IoT könnten AAS einerseits direkt aus PLM-Einträgen generiert und andererseits von Komponentenherstellern bereitgestellte AAS in das System übernommen werden. 
Es wird empfohlen, dass groninger den geplanten Umstieg gezielt dazu nutzt, die Möglichkeiten einer Integration der AAS in das PLM-System zu erproben, um so die Grundlage für eine skalierbare Erstellung digitaler Zwillinge zu schaffen.

Voraussetzung für eine solche Vorgehensweise ist, dass Zulieferer eigene \acs{aas} in ausreichendem Umfang bereitstellen. 
Da der Standard bislang erst von wenigen Unternehmen produktiv eingesetzt wird, stehen entsprechende \acs{aas} derzeit allerdings nur begrenzt zur Verfügung. 
Es ist jedoch zu erwarten, dass sich die Verbreitung der AAS mit der fortschreitenden Digitalisierung in den kommenden Jahren deutlich erhöhen wird. 
Groninger sollte diesen Trend aktiv beobachten und gegebenenfalls frühzeitig Kooperationen mit relevanten Zulieferern anstreben, um die notwendige Datenbasis sicherzustellen.

Darüber hinaus hat die Arbeit gezeigt, dass eine Kopplung der AAS mit realen Maschinen grundsätzlich möglich ist, etwa zur Erfassung von Betriebs- oder Zustandsdaten. 
Für groninger wird jedoch empfohlen, hierfür vorrangig spezialisierte Internet of Things (\acs{iot})-Lösungen einzusetzen, beispielsweise im Bereich Condition Monitoring oder OEE.
Auch für datengetriebene Analysen sollte die AAS nicht als primärer Speicher dienen, sondern vielmehr als übergeordnete Plattform, die zentrale Metadaten und Schnittstellen für den Datenzugriff verwaltet und dadurch die Basis für solche Anwendungen bildet.

Hinsichtlich steigender Anforderungen an Transparenz, insbesondere mit Blick auf regulatorische Vorgaben wie den zukünftig verpflichtenden \acs{dpp}, bietet die AAS erhebliches Potenzial. 
Auch wenn der \acs{dpp} zunächst nur für ausgewählte Branchen verpflichtend eingeführt wird, ist davon auszugehen, dass die Anforderungen langfristig auch den Sondermaschinenbau erreichen werden. 
Wie in dieser Arbeit gezeigt, bietet die \acs{aas} hierfür eine geeignete Grundlage, um nachhaltigkeitsbezogene Informationen standardisiert abzubilden und gezielt für unterschiedliche Interessengruppen bereitzustellen. 
Es wird daher empfohlen, dass groninger die weitere Entwicklung des DPP eng verfolgt und sich frühzeitig strategisch darauf vorbereitet, die AAS als technologische Grundlage für eine künftige Umsetzung zu nutzen.

\subsection{Ausblick auf zukünftige Entwicklungen}



%%%%---------------------------------
% \section{Zusammenfassung und Ausblick}
% \label{sec:Zusammenfassung}

% \subsection{Zusammenfassung der Arbeit}
% Ziel dieser Arbeit war es, am Beispiel des Abfüll- und Verschließmoduls der robocell-Linie das Potenzial der Asset Administration Shell (\acs{aas}) für die Modellierung und Nutzung eines digitalen Zwillings zu untersuchen.  
% Dazu wurden zunächst theoretische Grundlagen betrachtet und die eingesetzten Werkzeuge vorgestellt. Im praktischen Teil erfolgte die prototypische Implementierung einer Haupt-\acs{aas}, die sowohl statische als auch dynamische Submodelle umfasst und über OPC~UA mit realen Prozessdaten verknüpft ist. Ergänzend wurde ein \acs{ki}-basiertes Verfahren zur Anomalieerkennung entwickelt.  
% Darüber hinaus wurden zwei praxisnahe Anwendungsfälle exemplarisch umgesetzt: zum einen die Abbildung eines digitalen Produktpasses einschließlich PCF-Berechnung und differenzierter Zugriffsrechte, zum anderen die automatisierte Generierung von \acs{aas}.  
% Die Ergebnisse zeigen, dass die \acs{aas} ein vielversprechendes Konzept zur interoperablen und standardisierten Beschreibung von Maschinen und Komponenten darstellt. Insbesondere die Erweiterbarkeit durch Submodelle ermöglicht eine flexible Anpassung an unterschiedliche Anforderungen. Zugleich wurde deutlich, dass aktuelle Tools wie Eclipse BaSyx und der Package Explorer zwar eine wertvolle Basis bilden, in ihrer Praxistauglichkeit jedoch noch Einschränkungen aufweisen.

% \subsection{Handlungsempfehlung für groninger}
% Für groninger bietet der Einsatz der \acs{aas} eine vielversprechende Perspektive, um Transparenz, Nachverfolgbarkeit und Interoperabilität im Maschinen- und Anlagenbau zu stärken.  
% Kurzfristig empfiehlt es sich, den Ansatz in Pilotprojekten weiter zu erproben, um Erfahrungen mit Modellierung, Integration und Toolunterstützung zu sammeln. Mittelfristig sollte geprüft werden, wie die \acs{aas} in bestehende Systeme wie das PLM integriert und mit internen Datenmodellen verknüpft werden kann.  
% Besonderes Augenmerk sollte dabei auf regulatorische Entwicklungen gelegt werden. Der digitale Produktpass wird in den kommenden Jahren verpflichtend, sodass eine frühzeitige Auseinandersetzung mit entsprechenden Submodellen (z.\,B. Carbon Footprint) strategische Vorteile bietet. Auch die Nutzung von KI-Verfahren zur Zustandsüberwachung und Optimierung eröffnet Potenziale für Effizienzsteigerungen, die in Zukunft stärker verfolgt werden sollten.

% \subsection{Ausblick auf zukünftige Entwicklungen}
% Die weitere Entwicklung der \acs{aas} wird maßgeblich durch die Arbeit von Standardisierungsgremien wie der \acs{idta} geprägt. Neue Submodelle, eine stärkere Toolunterstützung sowie die Ausweitung auf gesamte Lieferketten werden die Einsatzmöglichkeiten in den kommenden Jahren erweitern.  
% Langfristig ist davon auszugehen, dass die \acs{aas} eine zentrale Rolle in digitalen Ökosystemen einnimmt und damit nicht nur einzelne Maschinen, sondern komplette Produktionsanlagen und Supply Chains digital abbildet. In diesem Kontext gewinnen auch Ansätze wie rollen- oder attributbasierte Zugriffskontrolle, erweiterte KI-Methoden zur Prozessanalyse sowie die Einbindung von Nachhaltigkeitskennzahlen zunehmend an Bedeutung.  
% Damit eröffnen sich für Unternehmen, die frühzeitig Erfahrungen mit der \acs{aas} sammeln, erhebliche Chancen, um künftige regulatorische Anforderungen zu erfüllen, interne Prozesse zu optimieren und sich technologisch im Wettbewerb zu positionieren.
