\newpage
\section{Zusammenfassung und Ausblick}
\label{sec:Zusammenfassung}
In diesem Kapitel werden zunächst die zentralen Ergebnisse der Arbeit zusammengefasst.
Daraufhin wird eine Handlungsempfehlung für groninger abgeleitet, die den möglichen Einsatz und Nutzen der \acs{aas} für das Unternehmen bewertet.
Abschließend folgt ein Ausblick auf zukünftige Weiterentwicklungen.

\subsection{Zusammenfassung der Arbeit}
Ausgehend von den theoretischen Grundlagen wurde ein Demonstrator entwickelt, mit dem sich die Umsetzung digitaler Zwillinge auf Basis der \acs{aas} erproben ließ.
Anhand statischer Submodelle wurde gezeigt, wie Stammdaten einer Maschine interoperabel und nachvollziehbar in der \acs{aas} abgebildet werden können.
Darauf aufbauend folgte eine dynamische Erweiterung, bei der die \acs{aas} als Typ-2-Ausprägung in einer Laufzeitumgebung bereitgestellt wurde und die Integration von Prozess -und Zeitreihendaten ermöglichte.
Dadurch entstand ein vollwertiger digitaler Zwilling, der nicht nur statische Informationen bereitstellt, sondern auch den aktuellen Betriebszustand der Maschine wiederspiegelt.

Ferner wurde ein \acs{ki}-Modell zur Anomalieerkennung konzipiert und prototypisch umgesetzt, das auf einem Autoencoder basiert und Temperaturdaten nutzt, die im digitalen Zwilling enthalten sind.
In Tests mit synthetisch erzeugten Anomalien gelang eine zuverlässige Erkennung, womit das Potenzial zur Optimierung der Anlagenverfügbarkeit und zur Reduzierung ungeplannter Stillstandzeiten aufgezeigt wurde.

Darüber hinaus wurden zwei praxisnahe Anwenungsfälle realisiert.
Der Anwendungsfall des \acs{dpp} zeigte, dass Nachhaltigkeitsinformationen in Form von \acs{pcf}-Werten transparent abgebildet und über Eclipse BaSyx rollenbasiert bereitgestellt werden können.
Damit wurde bestätigt, dass die \acs{aas} eine vielversprechende Grundlage für die praktische Umsetzung des \acs{dpp} gemäß dem von \acs{zvei} und \acs{idta} entwickelten Konzept des \acs{dpp40} ist.
Ergänzend ließ sich mit dem Anwendungsfall der automatisierten Generierung von \acs{aas} demonstrieren, wie sich der manuelle Modellierungsaufwand einzelner Instanzen verringern lässt und dadurch die effiziente Erstellung zahlreicher Instanzen in der industriellen Praxis möglich wäre.

\newpage
\subsection{Handlungsempfehlung für groninger}

Für groninger ergibt sich aus den in dieser Arbeit gewonnenen Erkenntnissen, dass die AAS eine geeignete Möglichkeit bietet, alle relevanten Maschinendaten gebündelt, interoperabel und maschinenlesbar bereitzustellen. 
Dadurch entsteht eine konsistente Informationsbasis, die sowohl intern genutzt als auch standardisiert an Partner weitergegeben werden kann. 
So könnte die AAS einer Maschine beispielsweise problemlos an einen Betreiber übermittelt werden, wenn dieser einen eigenen digitalen Zwilling erstellen möchte.

In der praktischen Anwendung erweist sich der Einsatz der AAS allerdings noch als nicht optimal, da das \acs{plm}-System Agile keine direkte Integration unterstützt. 
Die Erstellung einzelner \acs{aas} müsste daher gegenwärtig überwiegend manuell erfolgen, was mit erheblichem Aufwand verbunden wäre. 
Mittelfristig könnte sich dies mit dem geplanten Umstieg auf die CIM Database von CONTACT Software ändern.
Mit der Erweiterung CONTACT Elements for IoT eröffnen sich dort Möglichkeiten, \acs{aas} automatisiert zu erzeugen und einzubinden. 
Einerseits könnten \acs{aas} direkt aus \acs{plm}-Einträgen generiert und andererseits von Komponentenherstellern bereitgestellte \acs{aas} in das System übernommen werden. 
In Kombination entstünde so die Grundlage für eine weitgehend automatisierte Erstellung digitaler Zwillinge.

Voraussetzung für eine solche Vorgehensweise ist, dass Zulieferer eigene \acs{aas} in ausreichendem Umfang bereitstellen. 
Da der Standard bislang nur begrenzt verbreitet ist und erst von wenigen Unternehmen produktiv eingesetzt wird, stehen entsprechende \acs{aas} derzeit allerdings nur selten zur Verfügung. 
Es ist jedoch davon auszugehen, dass sich dieser Zustand mit der fortschreitenden Digitalisierung in den kommenden Jahren ändern wird.


Mittelfristig, mit dem geplanten Umstieg auf Contact Elements, 
+ kurzfristig nicht optimal
+ PLM-System ist nicht dafür ausgelget mit der AAS zu arbeiten - in Zukunft wird aber auf Contact Elements umgestiegen dann könnte es durchaus sinnn machen
+ Daten interoperabel Bereitstellen allles an einem ort gebündelt zum Beispiel wenn Betreiber eigenen Digitalen Zwilling, dann aber mit 3-D Simulationen
+ Iwan wird der digitale Produktpass kommen, dafür wäre es gut geeignet

\newpage
\subsection{Ausblick auf zukünftige Entwicklungen}



%%%%---------------------------------
% \section{Zusammenfassung und Ausblick}
% \label{sec:Zusammenfassung}

% \subsection{Zusammenfassung der Arbeit}
% Ziel dieser Arbeit war es, am Beispiel des Abfüll- und Verschließmoduls der robocell-Linie das Potenzial der Asset Administration Shell (\acs{aas}) für die Modellierung und Nutzung eines digitalen Zwillings zu untersuchen.  
% Dazu wurden zunächst theoretische Grundlagen betrachtet und die eingesetzten Werkzeuge vorgestellt. Im praktischen Teil erfolgte die prototypische Implementierung einer Haupt-\acs{aas}, die sowohl statische als auch dynamische Submodelle umfasst und über OPC~UA mit realen Prozessdaten verknüpft ist. Ergänzend wurde ein \acs{ki}-basiertes Verfahren zur Anomalieerkennung entwickelt.  
% Darüber hinaus wurden zwei praxisnahe Anwendungsfälle exemplarisch umgesetzt: zum einen die Abbildung eines digitalen Produktpasses einschließlich PCF-Berechnung und differenzierter Zugriffsrechte, zum anderen die automatisierte Generierung von \acs{aas}.  
% Die Ergebnisse zeigen, dass die \acs{aas} ein vielversprechendes Konzept zur interoperablen und standardisierten Beschreibung von Maschinen und Komponenten darstellt. Insbesondere die Erweiterbarkeit durch Submodelle ermöglicht eine flexible Anpassung an unterschiedliche Anforderungen. Zugleich wurde deutlich, dass aktuelle Tools wie Eclipse BaSyx und der Package Explorer zwar eine wertvolle Basis bilden, in ihrer Praxistauglichkeit jedoch noch Einschränkungen aufweisen.

% \subsection{Handlungsempfehlung für groninger}
% Für groninger bietet der Einsatz der \acs{aas} eine vielversprechende Perspektive, um Transparenz, Nachverfolgbarkeit und Interoperabilität im Maschinen- und Anlagenbau zu stärken.  
% Kurzfristig empfiehlt es sich, den Ansatz in Pilotprojekten weiter zu erproben, um Erfahrungen mit Modellierung, Integration und Toolunterstützung zu sammeln. Mittelfristig sollte geprüft werden, wie die \acs{aas} in bestehende Systeme wie das PLM integriert und mit internen Datenmodellen verknüpft werden kann.  
% Besonderes Augenmerk sollte dabei auf regulatorische Entwicklungen gelegt werden. Der digitale Produktpass wird in den kommenden Jahren verpflichtend, sodass eine frühzeitige Auseinandersetzung mit entsprechenden Submodellen (z.\,B. Carbon Footprint) strategische Vorteile bietet. Auch die Nutzung von KI-Verfahren zur Zustandsüberwachung und Optimierung eröffnet Potenziale für Effizienzsteigerungen, die in Zukunft stärker verfolgt werden sollten.

% \subsection{Ausblick auf zukünftige Entwicklungen}
% Die weitere Entwicklung der \acs{aas} wird maßgeblich durch die Arbeit von Standardisierungsgremien wie der \acs{idta} geprägt. Neue Submodelle, eine stärkere Toolunterstützung sowie die Ausweitung auf gesamte Lieferketten werden die Einsatzmöglichkeiten in den kommenden Jahren erweitern.  
% Langfristig ist davon auszugehen, dass die \acs{aas} eine zentrale Rolle in digitalen Ökosystemen einnimmt und damit nicht nur einzelne Maschinen, sondern komplette Produktionsanlagen und Supply Chains digital abbildet. In diesem Kontext gewinnen auch Ansätze wie rollen- oder attributbasierte Zugriffskontrolle, erweiterte KI-Methoden zur Prozessanalyse sowie die Einbindung von Nachhaltigkeitskennzahlen zunehmend an Bedeutung.  
% Damit eröffnen sich für Unternehmen, die frühzeitig Erfahrungen mit der \acs{aas} sammeln, erhebliche Chancen, um künftige regulatorische Anforderungen zu erfüllen, interne Prozesse zu optimieren und sich technologisch im Wettbewerb zu positionieren.
