\newpage
\section{Zusammenfassung und Ausblick}
\label{sec:Zusammenfassung}

Auf Grundlage der in den vorhergehenden Kapiteln dargestellten Ergebnisse werden im Folgenden die zentralen Erkenntnisse der Arbeit zusammengefasst. 
Daraufhin wird eine Handlungsempfehlung für groninger abgeleitet, die den möglichen Einsatz und Nutzen der \acs{aas} bewertet. 
Den Abschluss bildet ein Ausblick, der mögliche Weiterentwicklungen und künftige Ergänzungen dieser Arbeit aufzeigt.

\subsection{Zusammenfassung der Arbeit}

Ausgehend von den theoretischen Grundlagen wurde ein Demonstrator entwickelt, mit dem sich die Umsetzung digitaler Zwillinge auf Basis der \acs{aas} erproben ließ.
Anhand statischer Submodelle wurde gezeigt, wie Stammdaten einer Maschine interoperabel und nachvollziehbar in der \acs{aas} abgebildet werden können.
Darauf aufbauend folgte eine dynamische Erweiterung, bei der die \acs{aas} als Typ-2-Ausprägung in einer Laufzeitumgebung bereitgestellt wurde und die Integration von Prozess- und Zeitreihendaten ermöglichte.
Dadurch entstand ein vollwertiger digitaler Zwilling, der nicht nur statische Informationen bereitstellt, sondern auch den aktuellen Betriebszustand der Maschine widerspiegelt.

Ferner wurde ein \acs{ki}-Modell zur Anomalieerkennung konzipiert und prototypisch umgesetzt, das auf einem Autoencoder basiert und Temperaturdaten nutzt, die im digitalen Zwilling enthalten sind.
In Tests mit synthetisch erzeugten Anomalien gelang eine zuverlässige Erkennung, womit das Potenzial zur Optimierung der Anlagenverfügbarkeit und zur Reduzierung ungeplanter Stillstandzeiten aufgezeigt wurde.

Darüber hinaus wurden zwei praxisnahe Anwendungsfälle realisiert.
Der Anwendungsfall des \acs{dpp} zeigte, dass Nachhaltigkeitsinformationen in Form von \acs{pcf}-Werten transparent abgebildet und über Eclipse BaSyx rollenbasiert bereitgestellt werden können.
Dies bestätigte, dass die \acs{aas} eine vielversprechende Grundlage für die praktische Umsetzung des \acs{dpp} gemäß dem von \acs{zvei} und \acs{idta} entwickelten Konzept des \acs{dpp40} ist.
Ergänzend ließ sich mit dem Anwendungsfall der automatisierten Generierung von \acs{aas} demonstrieren, wie sich der manuelle Modellierungsaufwand reduzieren lässt und dadurch die effiziente Erstellung von \acs{aas}-Instanzen in der industriellen Praxis möglich wäre.

Abschließend wurden die eingesetzten Werkzeuge evaluiert. 
Dabei zeigte sich, dass sowohl der Package Explorer als auch Eclipse BaSyx eine zentrale Rolle bei der Modellierung und Bereitstellung von AAS einnehmen und maßgeblich zur erfolgreichen Umsetzung des Demonstrators und der Anwendungsfälle beigetragen haben, auch wenn in einzelnen Bereichen noch Optimierungspotenzial besteht.

\newpage
\subsection{Handlungsempfehlung für groninger}

Aus den in dieser Arbeit gewonnenen Erkenntnissen ergibt sich für groninger, dass die \acs{aas} ein geeignetes Instrument darstellt, um Maschinendaten strukturiert, interoperabel und maschinenlesbar bereitzustellen. 
Dadurch entsteht eine konsistente Informationsbasis, die sowohl intern genutzt als auch standardisiert an externe Partner weitergegeben werden kann. 
So ließe sich beispielsweise die \acs{aas} einer Maschine problemlos an einen Betreiber übermitteln, wenn dieser einen eigenen digitalen Zwilling aufbauen möchte.

In der praktischen Anwendung ist der Einsatz allerdings noch eingeschränkt, da das bestehende \acs{plm}-System Agile keine direkte Integration unterstützt. 
Die Erstellung einzelner \acs{aas} müsste derzeit manuell erfolgen und wäre mit hohem Aufwand verbunden.
Mit dem geplanten Umstieg auf die CIM Database von CONTACT Software eröffnen sich für groninger jedoch mittelfristig aussichtsreiche Perspektiven, diesen Prozess deutlich zu vereinfachen. 

Über CONTACT Elements for IoT \cite{CONTACT} könnten \acs{aas} einerseits direkt aus \acs{plm}-Einträgen generiert und andererseits von Komponentenherstellern bereitgestellte \acs{aas} in das System übernommen werden. 
Es wird empfohlen, dass groninger den geplanten Umstieg gezielt nutzt, um Integrationsmöglichkeiten zu erproben und so die Grundlage für eine skalierbare Erstellung digitaler Zwillinge zu schaffen.

Voraussetzung für eine solche Vorgehensweise ist, dass Zulieferer eigene \acs{aas} in ausreichendem Umfang bereitstellen. 
Da der Standard bislang erst von wenigen Unternehmen produktiv eingesetzt wird, stehen entsprechende Modelle derzeit allerdings nur begrenzt zur Verfügung. 
Es ist jedoch zu erwarten, dass sich die Verbreitung der \acs{aas} mit der fortschreitenden Digitalisierung in den kommenden Jahren deutlich erhöhen wird. 
Groninger sollte diesen Trend aktiv beobachten und gegebenenfalls frühzeitig Kooperationen mit relevanten Zulieferern anstreben, um die notwendige Datenbasis sicherzustellen.

Darüber hinaus hat die Arbeit gezeigt, dass eine Kopplung mit realen Maschinen grundsätzlich möglich ist, etwa zur Erfassung von Betriebs- oder Zustandsdaten. 
Für groninger wird jedoch empfohlen, hierfür vorrangig spezialisierte Internet of Things (\acs{iot})-Lösungen einzusetzen, beispielsweise im Bereich Condition Monitoring oder OEE.
Auch für datengetriebene Analysen sollte die \acs{aas} nicht als primärer Speicher dienen, sondern vielmehr als übergeordnete Plattform, die zentrale Metadaten und Schnittstellen für den Datenzugriff verwaltet und dadurch die Basis für solche Anwendungen bilden könnte.

Hinsichtlich steigender Anforderungen an Transparenz, insbesondere mit Blick auf regulatorische Vorgaben wie den zukünftig verpflichtenden \acs{dpp}, bietet die \acs{aas} erhebliches Potenzial. 
Auch wenn dieser zunächst nur für ausgewählte Branchen verpflichtend eingeführt wird, ist davon auszugehen, dass die Anforderungen langfristig auch den Sondermaschinenbau erreichen. 
Wie in dieser Arbeit gezeigt, bietet die \acs{aas} hierfür eine vielversprechende Lösung, um nachhaltigkeitsbezogene Informationen standardisiert abzubilden und gezielt unterschiedlichen Interessengruppen bereitzustellen. 
Daher wird empfohlen, dass 
\linebreak
groninger die weitere Entwicklung des \acs{dpp} eng verfolgt und sich frühzeitig darauf vorbereitet, die \acs{aas} als technologische Grundlage für eine künftige Umsetzung einzusetzen.

\subsection{Ausblick auf zukünftige Entwicklungen}
Der entwickelte Demonstrator hat gezeigt, dass die \acs{aas} bereits heute eine tragfähige Basis für die Erstellung digitaler Zwillinge bildet. 
Eine reale Maschine wurde bislang jedoch nicht angebunden, sondern das Verhalten durch zwei Anwendungen nachgebildet. 
Ein nächster logischer Schritt bestünde deshalb darin, die in diesem Projekt umgesetzte Anbindung in realen Szenarien zu erproben und dabei technische Herausforderungen sowie praktische Grenzen zu untersuchen. 

Ergänzend könnte die Kopplung der \acs{aas} mit Steuerungs- und Engineering-Umge\-bungen neue Potenziale eröffnen. 
So wäre es beispielsweise denkbar, \ac{plc}-Code automatisiert in die \acs{aas} zu übertragen und dort zu verwalten, wodurch eine engere Verzahnung von digitalem Zwilling und Steuerungsebene entstünde. 
Gleichzeitig könnten Entwicklungswerkzeuge profitieren, indem sie Daten direkt aus der \acs{aas} beziehen und damit den Engineering-Prozess effizient unterstützen. 
Ein entsprechender Ansatz wird bereits von Siemens mit dem TIA Portal verfolgt.

Zukünftige Entwicklungen sollten zudem verstärkt Sicherheitsaspekte berücksichtigen. 
Während in dieser Arbeit mit \acs{rbac} ein erster Ansatz umgesetzt wurde, erfordern produktive Szenarien, insbesondere in sicherheitskritischen Branchen wie der Pharmaindustrie, weitergehende Maßnahmen. 
Dazu zählen neben erweiterten Zugriffskontrollen auch die Verschlüsselung der Kommunikationswege sowie der Einsatz zusätzlicher Sicherheitsmechanismen, um sensible Informationen innerhalb der \acs{aas} zuverlässig zu schützen.

Des Weiteren eröffnet der Einsatz von Methoden der \acs{ki} vielversprechende Perspektiven.
Aufbauend auf dem in dieser Arbeit entwickelten Modell zur Anomalieerkennung könnten fortgeschrittene Verfahren, etwa im Bereich \acs{pm}, den Betrieb von Maschinen gezielt optimieren. 
Damit ließe sich der digitale Zwilling um wertvolle Prognosefunktionen erweitern. 
Zugleich eröffnet die \acs{aas} die Möglichkeit, \acs{ki}-Modelle über ihren gesamten Lebenszyklus hinweg in standardisierter Form zu verwalten. 
Erste \acsp{smt} wurden hierfür bereits veröffentlicht und könnten künftig als Grundlage dienen.
