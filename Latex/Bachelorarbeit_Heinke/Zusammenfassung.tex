\newpage
\section{Zusammenfassung und Ausblick}
\label{sec:Zusammenfassung}
\subsection{Zusammenfassung der Arbeit}
\subsection{Handlungsempfehlung für groninger}
+ kurzfristig nicht optimal
+ PLM-System ist nicht dafür ausgelget mit der AAS zu arbeiten - in Zukunft wird aber auf Contact Elements umgestiegen dann könnte es durchaus sinnn machen
+ Daten interoperabel Bereitstellen allles an einem ort gebündelt zum Beispiel wenn Betreiber eigenen Digitalen Zwilling, dann aber mit 3-D Simulationen
+ Iwan wird der digitale Produktpass kommen, dafür wäre es gut geeignet
\subsection{Ausblick auf zukünftige Entwicklungen}



%%%%---------------------------------
% \section{Zusammenfassung und Ausblick}
% \label{sec:Zusammenfassung}

% \subsection{Zusammenfassung der Arbeit}
% Ziel dieser Arbeit war es, am Beispiel des Abfüll- und Verschließmoduls der robocell-Linie das Potenzial der Asset Administration Shell (\acs{aas}) für die Modellierung und Nutzung eines digitalen Zwillings zu untersuchen.  
% Dazu wurden zunächst theoretische Grundlagen betrachtet und die eingesetzten Werkzeuge vorgestellt. Im praktischen Teil erfolgte die prototypische Implementierung einer Haupt-\acs{aas}, die sowohl statische als auch dynamische Submodelle umfasst und über OPC~UA mit realen Prozessdaten verknüpft ist. Ergänzend wurde ein \acs{ki}-basiertes Verfahren zur Anomalieerkennung entwickelt.  
% Darüber hinaus wurden zwei praxisnahe Anwendungsfälle exemplarisch umgesetzt: zum einen die Abbildung eines digitalen Produktpasses einschließlich PCF-Berechnung und differenzierter Zugriffsrechte, zum anderen die automatisierte Generierung von \acs{aas}.  
% Die Ergebnisse zeigen, dass die \acs{aas} ein vielversprechendes Konzept zur interoperablen und standardisierten Beschreibung von Maschinen und Komponenten darstellt. Insbesondere die Erweiterbarkeit durch Submodelle ermöglicht eine flexible Anpassung an unterschiedliche Anforderungen. Zugleich wurde deutlich, dass aktuelle Tools wie Eclipse BaSyx und der Package Explorer zwar eine wertvolle Basis bilden, in ihrer Praxistauglichkeit jedoch noch Einschränkungen aufweisen.

% \subsection{Handlungsempfehlung für groninger}
% Für groninger bietet der Einsatz der \acs{aas} eine vielversprechende Perspektive, um Transparenz, Nachverfolgbarkeit und Interoperabilität im Maschinen- und Anlagenbau zu stärken.  
% Kurzfristig empfiehlt es sich, den Ansatz in Pilotprojekten weiter zu erproben, um Erfahrungen mit Modellierung, Integration und Toolunterstützung zu sammeln. Mittelfristig sollte geprüft werden, wie die \acs{aas} in bestehende Systeme wie das PLM integriert und mit internen Datenmodellen verknüpft werden kann.  
% Besonderes Augenmerk sollte dabei auf regulatorische Entwicklungen gelegt werden. Der digitale Produktpass wird in den kommenden Jahren verpflichtend, sodass eine frühzeitige Auseinandersetzung mit entsprechenden Submodellen (z.\,B. Carbon Footprint) strategische Vorteile bietet. Auch die Nutzung von KI-Verfahren zur Zustandsüberwachung und Optimierung eröffnet Potenziale für Effizienzsteigerungen, die in Zukunft stärker verfolgt werden sollten.

% \subsection{Ausblick auf zukünftige Entwicklungen}
% Die weitere Entwicklung der \acs{aas} wird maßgeblich durch die Arbeit von Standardisierungsgremien wie der \acs{idta} geprägt. Neue Submodelle, eine stärkere Toolunterstützung sowie die Ausweitung auf gesamte Lieferketten werden die Einsatzmöglichkeiten in den kommenden Jahren erweitern.  
% Langfristig ist davon auszugehen, dass die \acs{aas} eine zentrale Rolle in digitalen Ökosystemen einnimmt und damit nicht nur einzelne Maschinen, sondern komplette Produktionsanlagen und Supply Chains digital abbildet. In diesem Kontext gewinnen auch Ansätze wie rollen- oder attributbasierte Zugriffskontrolle, erweiterte KI-Methoden zur Prozessanalyse sowie die Einbindung von Nachhaltigkeitskennzahlen zunehmend an Bedeutung.  
% Damit eröffnen sich für Unternehmen, die frühzeitig Erfahrungen mit der \acs{aas} sammeln, erhebliche Chancen, um künftige regulatorische Anforderungen zu erfüllen, interne Prozesse zu optimieren und sich technologisch im Wettbewerb zu positionieren.
